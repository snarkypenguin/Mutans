% -*- outline-regexp: "%--* ";  -*-
\documentclass[a4]{article}
%- Packages
%% Unicode...
% \usepackage{ucs}
% \usepackage[utf8x]{inputenc}
% \usepackage{textcomp}

\usepackage{url,lineno}

%-- Mathematics
\usepackage{mathtools}
% \usepackage{accents}
\usepackage{amsmath}
% \usepackage{amsthm}
% \usepackage{amssymb}
% \usepackage{mathtools}
% \usepackage{enumitem}
\usepackage{natbib}
\usepackage{lmodern}

%-- Algorithms
\usepackage{algorithm}
% \usepackage[defblank]{paralist}
%% \usepackage{algorithmicx}
\usepackage{algpseudocode}

%-- Lists
\usepackage[ampersand]{easylist}

%-- Fonts
\usepackage[T1]{fontenc}
\usepackage{calligra}
\usepackage{chancery}


%- Flags, such as \linenumbers
%%% Uncomment to use
% \linenumbers


%- Fonts
\def\origrm{rmdefault}
\renewcommand{\rmdefault}{origrm}

\newcommand{\resetfonts}[0]{\fontencoding{\encodingdefault}\fontfamily{\familydefault}\fontseries{\seriesdefault}\fontshape{\shapedefault}\selectfont}
\def\calli#1{{\calligra #1}}
\def\chanc#1{{\renewcommand{\rmdefault}{pzc}\rm #1 \renewcommand{\rmdefault}{origrm}}}

%- Maths alphabets
\DeclareMathAlphabet{\mathsfsl}{OT1}{cmss}{m}{sl}
\DeclareMathAlphabet{\mathpzc}{OT1}{pzc}{m}{it}

%- Maths operators

%- Load extra configuration before the newcommands 
% \message{Loading Manifest ==========================================}
% \typeout{******* in manifest.tex}
%% \newcommand{\blockcomment}[1]{}
%% \newcommand{\UseFirstVersion}[2]{#1}
%% \newcommand{\UseSecondVersion}[2]{#2}
%% \newcommand{\UseNeitherVersion}[2]{}



\typeout{bungie jumping}
\iftoggle{usesMnSymbol}{
   \typeout{using MnSymbol}
   \newcommand{\BBN}[0]{\mathbb{N}}
   \newcommand{\BBNO}[0]{\mathbb{N}^0}
   \newcommand{\BBNI}[0]{\mathbb{N}^+}
   \newcommand{\mB}[1]{\mathbb{#1}}
   \newcommand{\dotcup}{\ensuremath{\mathaccent\cdot\cup}}
}{
   \typeout{Hack the barstud}
   \DeclareMathAlphabet{\mathoss}{OT1}{cmss}{m}{sl}
   \DeclareMathAlphabet{\mathsfsl}{OT1}{cmss}{m}{sl}
   \typeout{declaring BBN BBNO BBNI mB ranglebar langlebar dotcup}
   \newcommand{\BBN}[0]{\mathbb{N}}
   \newcommand{\BBNO}[0]{\mathbb{N}^0}
   \newcommand{\BBNI}[0]{\mathbb{N}^+}
   \newcommand{\mB}[1]{\mathbb{#1}}
   \newcommand{\ranglebar}{\lvert\rangle}
   \newcommand{\langlebar}{\langle\rvert}
   \newcommand{\dotcup}{\ensuremath{\mathaccent\cdot\cup}}
}

\newcommand{\mt}[1]{\mathtt{#1}}
\newcommand{\ms}[1]{\mathsf{#1}}
\newcommand{\mb}[1]{\boldsymbol{#1}}  %% from amsbsy
\newcommand{\mf}[1]{\mathfrak{#1}}
\newcommand{\mc}[1]{\mathpzc{#1}}

%%\newcommand{\mb}[1]{\mathbf{#1}}
%\newcommand{\mo}[1]{\mathocm{#1}} 


%\setmathfont{MnSymbol}
%\setmathfont[range={\llangle,\rrangle}]{XITS Math}

\typeout{set ops}
%\newcommand{\msdec}[1]{\ddot{#1}}
%\newcommand{\msdec}[1]{\dot{#1}}
\newcommand{\set}[1] {\mb{#1}}
%\newcommand{\msdec}[1]{\bar{#1}}
\newcommand{\msdec}[1]{\bar{#1}}
\newcommand{\mset}[1]{\mb{\msdec{#1}}}
\newcommand{\multiset}[1]{\mset{#1}}
\newcommand{\TN}[1]{\Phi(\overline{\mset{#1}})}

%% \newcommand{\zz}[3]{\mathrel{\substack{\text{{\Large S}}\\{#1}\in{#2}}{#3}}}
%% \newcommand{\szz}[3]{\mathrel{\substack{\text{{\Large S}}^\Sigma\\{#1}\in{#2}}{#3}}}
%% \newcommand{\Zz}[3]{\mathrel{\substack{\text{{\huge S}}\\{#1}\in{#2}}{#3}}}
%% \newcommand{\sZz}[3]{\mathrel{\substack{\text{{\huge S}}^\Sigma\\{#1}\in{#2}}{#3}}}
%% \newcommand{\ZZ}[3]{\mathrel{\substack{\text{{\Huge S}}\\{#1}\in{#2}}{#3}}}
%% \newcommand{\sZZ}[3]{\mathrel{\substack{\text{{\Huge S}}^\Sigma\\{#1}\in{#2}}{#3}}}

%% From tex.stackexchange:
%% Macro \vv breaks in a \typeout message, because \vv is not robust. \protect helps:

%% \documentclass{article}

%% \usepackage{esvect}
%% \usepackage[outline]{contour}

%% \begin{document}
%%   \contour{red}{$\protect\vv{aa}$}
%% \end{document}


\newcommand{\at}{\makeatletter @\makeatother}

\makeatletter
\newcommand{\interitemtext}[1]{%
\begin{list}{}
{\itemindent=0mm\labelsep=0mm
\labelwidth=0mm\leftmargin=0mm
\addtolength{\leftmargin}{-\@totalleftmargin}}
\item #1
\end{list}}
\makeatother

\newcommand{\Cite}[1]{\cite{#1}}

\DeclareMathOperator{\m}{m}
\DeclareMathOperator{\agg}{agg}
\DeclareMathOperator{\term}{term}
\DeclareMathOperator{\coeff}{coeff}
\DeclareMathOperator{\scalar}{scalar}
\DeclareMathOperator{\nonscalar}{nonscalar}
\DeclareMathOperator{\opt}{opt}
\DeclareMathOperator{\kernl}{kern}
\DeclareMathOperator{\compatible}{compatible}
\DeclareMathOperator{\depth}{depth}
\DeclareMathOperator{\supp}{supp}
\DeclareMathOperator{\fund}{fund}
\DeclareMathOperator{\gen}{}
\DeclareMathOperator{\trim}{trim}
\DeclareMathOperator{\prune}{prune}
\DeclareMathOperator{\labels}{labels}
\DeclareMathOperator{\dist}{d}
\DeclareMathOperator{\overlap}{overlap}
\DeclareMathOperator{\shadow}{shadow}
\DeclareMathOperator{\boundary}{bnd}
\DeclareMathOperator{\interior}{int}
\DeclareMathOperator{\excise}{excise}
%\DeclareMathOperator{\devi}{dev}

%\def\capcross{\mathrel{\mathchoice{\CAPCROSS}{\CAPCROSS}{\scriptsize\CAPCROSS}{\tiny\CAPCROSS}}}
%\def\CAPCROSS{{\setbox0\hbox{\cap}\rlap{\hbox to \wd0{\centerdot}}\box0}}

%% there are three overlap things: \llap \clap and \rlap

\def\SIM{{\setbox0\hbox{\cap}\rlap{\hbox\wd0{\sim}}\box0}}

\def\SDC{{\setbox0\hbox{\setminus}\rlap{\hbox\wd0{\approx}}\box0}}
\def\NTC{{\setbox0\hbox{\cap}\rlap{\hbox\wd0{\approx}}\box0}}



%%\newcommand\scaleobj[2]{\hstretch{#1}{\vstretch{#1}{#2}}}
%%\newcommand\Alpha{\scaleobj{1.5}{\alpha}}


\newcommand{\UpnotVp}[2]{\set{#1}\bbslash\set{#2}}
\newcommand{\UpandVp}[2]{\set{#1}\doublecap\set{#2}}

\DeclareMathOperator{\mie}{\ensuremath{\text{~i.e.~}}}
\DeclareMathOperator{\meg}{\ensuremath{\text{~e.g.~}}}
\DeclareMathOperator{\mst}{\ensuremath{\text{~s.t.~}}}
\DeclareMathOperator{\mand}{\ensuremath{\text{~and~}}}
\DeclareMathOperator{\mbut}{\ensuremath{\text{~but~}}}
\DeclareMathOperator{\mor}{\ensuremath{\text{~or~}}}
\DeclareMathOperator{\mnot}{\ensuremath{\text{~not~}}}
\DeclareMathOperator{\otherwise}{\ensuremath{\text{~otherwise~}}}
\DeclareMathOperator{\md}{d\!}

\DeclareMathOperator{\oop}{OutOp2}
\DeclareMathOperator{\iop}{InOp2}

\DeclareMathOperator{\ssim}{ssim}
\DeclareMathOperator{\relabel}{relabel}
\DeclareMathOperator{\mask}{mask}
\DeclareMathOperator{\nmask}{\overline{mask}}

\DeclareMathOperator{\mI}{mI}
\DeclareMathOperator{\mSG}{mSG}
\DeclareMathOperator{\mSL}{mSL}

\DeclareMathOperator{\drawf}{drawf}
\DeclareMathOperator{\lfwdf}{fwd}
\DeclareMathOperator{\linvf}{inv}
\DeclareMathOperator{\lscalarf}{scalar}

\newtheorem{notation}{Notation}[]
\newtheorem{definition}{Definition}[section]
\newtheorem{corollary}{Corollary}[section]
\newtheorem{proposition}{Proposition}[section]
\newtheorem{lemma}{Lemma}[section]
\newtheorem{theorem}{Theorem}[section]
\newtheorem{remark}{Remark}[section]
\newtheorem{example}{Example}[section]
\newtheorem{prototype}{Prototype}[section]

\newcommand{\draw}[1]{\drawf(#1)}
\newcommand{\lfwd}[1]{\lfwdf(#1)}
\newcommand{\linv}[1]{\linvf(#1)}
\newcommand{\lscalar}[1]{\lscalarf(#1)}

\newcommand{\LS}{\Lambda^{{}^{\Sigma}}}


\newcommand{\istate}{\emph{i}-state}
\newcommand{\istateC}{\emph{i}-state configuration}
\newcommand{\istateD}{\emph{i}-state distribution}
\newcommand{\pstate}{\emph{p}-state}

%{\mbox{\mathsurround=0pt \makebox[0pt][l]{\(\cap\)}\(\centerdot\)})

\newcommand{\Sim}[2]{{#1}\SIM{#2}}
\newcommand{\NSim}[2]{\overline{\Sim{#1}{#2}}}

%\newcommand{\marginnote}[1]{\marginpar{\scriptsize{#1}}}
%\newcommand{\marginnote}[1]{\marginpar{\footnotesize{#1}}}
%\newcommand{\marginnote}[1]{\marginpar{\small{#1}}}}
%\newcommand{\marginnote}[1]{\marginpar{#1}}

\newcommand{\marginnote}[1]{}

%\def\MSCUP{{\setbox0\hbox{\cup}\rlap{\hbox to \wd0{\dagger}}\box0}}
%\def\mscup{\mathrel{\mathchoice{\MSCUP}{\MSCUP}{\scriptsize\MSCUP}{\tiny\CUPPLUS}}}

\iftoggle{usesMnSymbol}{
\newcommand{\msetminus}{\bbslash}
\newcommand{\mscup}{\uplus}
}{
\newcommand{\msetminus}{\bbslash}
\newcommand{\mscup}{\uplus}
\newcommand{\cupdot}{\udot}
\newcommand{\cupplus}{\uplus}
\newcommand{\bigcupplus}{\biguplus}
}

\newcommand{\eqc}[1]{\left[#1\right]}
\newcommand{\TT}[1]{\texttt{{#1}}}
\newcommand{\TTC}[1]{\texttt}

%\newcommand{\capplus}{%
%    \setbox0\hbox{\cap}%
%    \rlap{\hbox to \wd0{\hss+\hss}}\box0
%}

% Some useful definitions...

\newcommand{\I}{\mathsf{I}}
\newcommand{\FillThisIn}[1]{{\textbf{\Large{#1}}}}
\newcommand{\textbsf}[1]{\textbf{\textsf{#1}}}

%\newcommand{\NOTE}[1]{}
\newcommand{\NOTE}[1]{\marginpar{\textbf{\Large #1}}}
\newcommand{\imarginnote}[1]{$\,\!$\\\noindent\vspace{-\baselineskip}\marginnote{#1}}

\newcommand{\WORK}[1]{{\LARGE{\textbf{Work here $\Downarrow$ {#1}}}}}

\newcommand{\FIX}[0]{\textbf{(Need better phrasing)}}
\newcommand{\MORE}[1]{\textbf{More here #1}}
\newcommand{\NRH}[1]{\textbf{(Need reference here #1)}}
\newcommand{\REFER}[1]{\textbf{(Need reference here #1)}}
\newcommand{\POINTS}[1]{\marginnote{#1}}
\newcommand{\NotHere}[1]{\textsf{#1}}

\newcommand{\HERE}{{\Huge{**** HERE ****}}}
\newcommand{\NB}[1]{\Large{\emph{#1}}}

\newcommand{\etal}[0]{\emph{et al.}}
%\newcommand{\etal}[0]{et al.}

\newcommand{\ie}[0]{\emph{i.e.}}
%\newcommand{\ie}[0]{i.e.}

\newcommand{\eg}{\emph{e.g.}}
%\newcommand{\eg}[0]{e.g.}

\newcommand{\tsup}[1]{\textsuperscript{#1}}
\newcommand{\tsub}[1]{\textsubscript{#1}}
\newcommand{\Defn}[0]{Def\tsup{n}}

\newcommand{\stsum}[1]{\sum_{\substack{#1}}}
\newcommand{\stSum}[1]{\sum_{\substack{#1}}}

\newcommand{\mstsum}[1]{\sum_{\mathclap{\substack{#1}}}}
\newcommand{\mstSum}[1]{\sum_{\mathclap{\substack{#1}}}}

\newcommand{\bdot}{%
\setbox0\hbox{\box}
\rlap{\hbox{ to \wd0{\hss\cdot\hss}}}\box0}

\newcommand{\bplus}{\boxplus}

%\newcommand{\tplus}{\maltese}
\newcommand{\tplus}{+}
\newcommand{\tdot}{\cdot}
\newcommand{\stplus}{\boxplus}
\newcommand{\stdot}[0]{\boxdot}

\newcommand{\haschild}[2]{{#1}\rightarrow{#2}}
\newcommand{\hasextn}[2]{{#1}\rightarrow{#2}}
\newcommand{\haschildren}[2]{{#1}\Rightarrow{#2}}
\newcommand{\hasextension}[2]{{#1}\Rightarrow{#2}}

%-- 

\newcommand{\REAL}{\mB{R}}
\newcommand{\COMPLEX}{\mB{C}}
\newcommand{\FIELD}{\mB{K}}

\newcommand{\TREAL}{\(\REAL\)}
\newcommand{\TCOMPLEX}{\(\COMPLEX\)}
\newcommand{\TFIELD}{\(\FIELD\)}

%-- 

\newcommand{\A}{\mc{A}}
\newcommand{\PLY}[1]{\FIELD[{\set{#1}}]}

\newcommand{\SDOMstar}{\set{T}^*}
\newcommand{\DOMstar}{\mb{T}^*}
\newcommand{\DOMstars}[1]{{\mb{T}^*}_\mc{#1}}
\newcommand{\TSDOMstar}{\(\SDOMstar\!\)}

\newcommand{\SDOM}{\set{T}}
\newcommand{\DOM}{\mb{T}}
\newcommand{\DOMs}[1]{{\mb{T}}_\mc{#1}}
\newcommand{\TSDOM}{\(\SDOM\!\)}


\newcommand{\devi}[2]{\Delta(#1,#2)}

%-- 

\newcommand{\TDOM}{\(\DOM\)}
%\newcommand{\TDOM}{\textbf{T}}
\newcommand{\TDOMs}[1]{\DOMs{#1}}

\newcommand{\xTree}[0]{$\eta$-Tree}
\newcommand{\xTrees}[0]{$\eta$-Trees}
\newcommand{\xtree}[0]{$\eta$-tree}
\newcommand{\xtrees}[0]{$\eta$-trees}


\newcommand{\poly}[1]{\mathsf{#1}}



\newcommand{\polynomialfactor}[3]{\Pi_{i=1}^{#3} {#1}_{i\/j}^{{#2}_{i\/j}}}

%%%%%%%%%%%%%%%%%%%%%%%%%%%%%%   a                 k    a                             x   e   n
\newcommand{\explicitpoly}[5]{{#1}_0 + \sum_{j=1}^{#4} {#1}_j \bigl(\polynomialfactor{#2}{#3}{#5}\bigr)}


\newcommand{\node}[1]{\,\mb{#1}}
\newcommand{\onetree}{\node{\I}}
\newcommand{\zerotree}{\node{\mathsf{O}}}
\newcommand{\tnulltree}{\(\nulltree\)}
\newcommand{\nulltree}{\node{\mathsf{O}}}
\iftoggle{frakok}{
   \newcommand{\nullspace}{\mf{O}}
}{
   \DeclareMathAlphabet{\mathfrak}{OT1}{}{m}{sl}
   \newcommand{\nullspace}{\underline{\overline{\mc{O}}}}
}

\newcommand{\nel}[2]{{\node{#1}}_{{}_{#2}}}
\newcommand{\nv}[1]{\nel{#1}{v}}
\newcommand{\extn}[1]{\nel{#1}{\set{E}}}
%\newcommand{\child}[1]{\nel{#1}{\msdec{\set{C}}}}
\newcommand{\child}[1]{\nel{#1}{\set{C}}}
\newcommand{\nlabel}[1]{\nel{#1}{\mathsf{P}}}
\newcommand{\nslabel}[2]{\nel{#1}{\mathsf{P}_{#2}}}
\newcommand{\ninertia}[1]{\nel{#1}{r}}

\newcommand{\Zerotree}{\NodeIII{0}{0}{\emptyset}}

\newcommand{\cross}{\times}
\newcommand{\crossproduct}{\otimes}
\newcommand{\Xproduct}[2]{{#1}\crossproduct{#2}}

\newcommand{\tnode}[1]{\(\node{#1}\)}
\newcommand{\tset}[1]{\(\set{#1}\)}
\newcommand{\tonetree}{\(\onetree\)}
\newcommand{\tzerotree}{\(\zerotree\)}
\newcommand{\tnullspace}{\(\nullspace\)}
\newcommand{\tnel}[2]{\(\nel{#1}{#2}\)}
\newcommand{\textn}[1]{\(\extn{#1}\)}
\newcommand{\tchild}[1]{\(\child{#1}\)}
\newcommand{\tnlabel}[1]{\(\nlabel{#1}\)}
\newcommand{\tnv}[1]{\(\nv{#1}\)}
\newcommand{\tninertia}[1]{\(\ninertia{#1}\)}
\newcommand{\tnK}[1]{\(\nK{#1}\)}
\newcommand{\tnI}[1]{\(\nI{#1}\)}
\newcommand{\tnM}[1]{\(\nM{#1}\)}

\newcommand{\nlabels}[1] {\mb{L}(\node{#1})}
\newcommand{\notnlabels}[1] {\overline{\mb{L}}(\node{#1})}

\newcommand{\srestrictedto}[2]{\set{#1}\vert_{\nlabels{#2}}}
\newcommand{\nsrestrictedto}[2]{\set{#1}\vert_{\notnlabels{#2}}}
\newcommand{\restrictedto}[2]{\node{#1}\vert_{\nlabels{#2}}}
\newcommand{\nrestrictedto}[2]{\node{#1}\vert_{\notnlabels{#2}}}

\newcommand{\childO}[5]{\bigl(\nrestrictedto{#1}{#2}\cup\nrestrictedto{#2}{#1}\cup\{\node{#3}{#5}\node{#4}:\node{#3}\in\restrictedto{#1}{#2}\mand\node{#4}\in\restrictedto{#2}{#1}\mand\nlabel{#3}=\nlabel{#4}\}\bigr)\setminus\{\zerotree\}}
\newcommand{\childOb}[5]{\bigl(\nrestrictedto{#1}{#2}\cup\nrestrictedto{#2}{#1}\\
&\qquad\cup\{\node{#3}{#5}\node{#4}:\node{#3}\in\restrictedto{#1}{#2}\mand\node{#4}\in\restrictedto{#2}{#1}\mand\nlabel{r}=\nlabel{s}\}\bigr)\setminus\{\zerotree\}}

\newcommand{\childObR}[5]{\bigl(\nrestrictedto{#1}{#2}\cup\nrestrictedto{#2}{#1}\\
&\qquad\cup\{\node{#3}{#5}\node{#4}:\node{#3}\in\restrictedto{#1}{#2}\mand\node{#4}\in\restrictedto{#2}{#1}\mand\nlabel{r}=\nlabel{s}\}\bigr)\setminus\{\zerotree\}}
\newcommand{\childObL}[5]{\bigl(\nrestrictedto{#1}{#2}\cup\nrestrictedto{#2}{#1}&\\
\qquad\cup\{\node{#3}{#5}\node{#4}:\node{#3}\in\restrictedto{#1}{#2}\mand\node{#4}\in\restrictedto{#2}{#1}\mand\nlabel{r}=\nlabel{s}\}\bigr)\setminus\{\zerotree\}}

\newcommand{\extnO}[5]{\bigl(\nrestrictedto{#1}{#2}\cup\nrestrictedto{#2}{#1}\cup\{\node{#3}{#5}\node{#4}:\node{#3}\in\restrictedto{#1}{#2}\mand\node{#4}\in\restrictedto{#2}{#1}\mand\nlabel{#3}=\nlabel{#4}\}\bigr)\setminus\{\zerotree\}}
\newcommand{\extnOb}[5]{\bigl(\nrestrictedto{#1}{#2}\cup\nrestrictedto{#2}{#1} \\
&\qquad\cup\{\node{#3}{#5}\node{#4}:\node{#3}\in\restrictedto{#1}{#2}\mand\node{#4}\in\restrictedto{#2}{#1}\mand\nlabel{r}=\nlabel{s}\}\bigr)\setminus\{\zerotree\}}

\newcommand{\extnObR}[5]{\bigl(\nrestrictedto{#1}{#2}\cup\nrestrictedto{#2}{#1} \\
&\qquad\cup\{\node{#3}{#5}\node{#4}:\node{#3}\in\restrictedto{#1}{#2}\mand\node{#4}\in\restrictedto{#2}{#1}\mand\nlabel{r}=\nlabel{s}\}\bigr)\setminus\{\zerotree\}}
\newcommand{\extnObL}[5]{\bigl(\nrestrictedto{#1}{#2}\cup\nrestrictedto{#2}{#1}&\\
\qquad\cup\{\node{#3}{#5}\node{#4}:\node{#3}\in\restrictedto{#1}{#2}\mand\node{#4}\in\restrictedto{#2}{#1}\mand\nlabel{r}=\nlabel{s}\}\bigr)\setminus\{\zerotree\}}

\newcommand{\sv}[1]{\ensuremath{\scalar(\nlabel{#1})}}

\newcommand{\cring}{ring}
\newcommand{\rng}{rng}

\newcommand{\RNGI}[1]{\breve{#1}}
\newcommand{\RNG}[1]{\check{#1}}

\UseSecondVersion{
\newcommand{\CRING}[1]{\RNGI{\mb{T}}}
}{
\newcommand{\CRING}[1]{\RNG{\mb{T}}}
}

\newcommand{\STRUCT}{field}
\newcommand{\Struct}{Field}
\newcommand{\struct}{field}

\newcommand{\DOMQ}{\RNG{\mb{T}}}
\newcommand{\DOMQs}[1]{\RNG{{\mb{T}}_\mc{#1}}}

\newcommand{\TDOMQ}{\(\DOMQ\)}
\newcommand{\TDOMQs}[1]{\(\DOMQs{#1}\)}

\newcommand{\qnode}[1]{\,\RNG{\mb{#1}}}
\newcommand{\qonetree}{\qnode{\I}}
\newcommand{\qzerotree}{\qnode{\mathsf{O}}}
\newcommand{\qnulltree}{\qnode{\mathsf{O}}}
\newcommand{\qel}[2]{{\qnode{#1}}{{}_{_{#2}}}}
\newcommand{\qextn}[1]{\qel{#1}{\set{E}}}
\newcommand{\qchild}[1]{\qel{#1}{\msdec{\set{C}}}}
\newcommand{\qlabel}[1]{\qel{#1}{\mathsf{P}}}
\newcommand{\qv}[1]{\qel{#1}{\nu}}
\newcommand{\qinertia}[1]{\qel{#1}{r}}
\newcommand{\qK}[1]{\qel{#1}{K}}
\newcommand{\qI}[1]{\qel{#1}{I}}
\newcommand{\qM}[1]{\qel{#1}{M}}

\newcommand{\tqnode}[1]{\(\RNG{\node{#1}}\)}
\newcommand{\tqonetree}{\(\onetree\)}
\newcommand{\tqzerotree}{\(\qnode{\mathsf{O}}\)}
\newcommand{\tqnulltree}{\(\qnode{\mathsf{O}}\)}
\newcommand{\tqel}[2]{\({\qnode{#1}}_{#2}\)}
\newcommand{\tqextn}[1]{\(\qel{#1}{\set{E}}\)}
\newcommand{\tqchild}[1]{\(\qchild{#1}\)}
\newcommand{\tqlabel}[1]{\(\qel{#1}{\mathsf{P}}\)}
\newcommand{\tqv}[1]{\(\qel{#1}{\nu}\)}
\newcommand{\tqinertia}[1]{\(\qel{#1}{r}\)}
\newcommand{\tqK}[1]{\(\qel{#1}{K}\)}
\newcommand{\tqI}[1]{\(\qel{#1}{I}\)}
\newcommand{\tqM}[1]{\(\qel{#1}{M}\)}

\newcommand{\qlabels}[1] {\mb{L}(\qnode{#1})}
\newcommand{\notqlabels}[1] {\overline{\mb{L}}(\qnode{#1})}

\newcommand{\srestrictedtoq}[2]{\set{#1}\vert_{\qlabels{#2}}}
\newcommand{\nsrestrictedtoq}[2]{\set{#1}\vert_{\notqlabels{#2}}}
\newcommand{\restrictedtoq}[2]{\qnode{#1}\vert_{\qlabels{#2}}}
\newcommand{\nrestrictedtoq}[2]{\qnode{#1}\vert_{\notqlabels{#2}}}

\newcommand{\extnOq}[5]{\bigl(\nrestrictedtoq{#1}{#2}\cup\nrestrictedtoq{#2}{#1}\cup\{\qnode{#3}{#5}\qnode{#4}:\qnode{#3}\in\restrictedtoq{#1}{#2}\mand\qnode{#4}\in\restrictedtoq{#2}{#1}\mand\qlabel{#3}=\qlabel{#4}\}\bigr)\setminus\{\zerotree\}}
\newcommand{\extnObq}[5]{\bigl(\nrestrictedtoq{#1}{#2}\cup\nrestrictedtoq{#2}{#1} \\
&\qquad\cup\{\qnode{#3}{#5}\qnode{#4}:\qnode{#3}\in\restrictedtoq{#1}{#2}\mand\qnode{#4}\in\restrictedtoq{#2}{#1}\mand\qlabel{r}=\qlabel{s}\}\bigr)\setminus\{\zerotree\}}

\newcommand{\childOq}[5]{\bigl(\nrestrictedtoq{#1}{#2}\cup\nrestrictedtoq{#2}{#1}\cup\{\qnode{#3}{#5}\qnode{#4}:\qnode{#3}\in\restrictedtoq{#1}{#2}\mand\qnode{#4}\in\restrictedtoq{#2}{#1}\mand\qlabel{#3}\sim\qlabel{#4}\}\bigr)\setminus\{\zerotree\}}
\newcommand{\childObq}[5]{\bigl(\nrestrictedtoq{#1}{#2}\cup\nrestrictedtoq{#2}{#1} \\
&\qquad\cup\{\qnode{#3}{#5}\qnode{#4}:\qnode{#3}\in\restrictedtoq{#1}{#2}\mand\qnode{#4}\in\restrictedtoq{#2}{#1}\mand\qlabel{r}\sim\qlabel{s}\}\bigr)\setminus\{\zerotree\}}

%-- 

\newcommand{\DOMR}{\RNG{\mb{T}}}
\newcommand{\DOMRs}[1]{\RNG{{\mb{T}^\star}_\mc{#1}}}

\newcommand{\TDOMR}{\(\DOMR\)}
\newcommand{\TDOMRs}[1]{\(\DOMRs{#1}\)}

\newcommand{\one}{\rnode{\jmath}}

\newcommand{\rnode}[1]{\,\RNG{\mb{#1}}}
\newcommand{\ronetree}{\rnode{\I}}
\newcommand{\rzerotree}{\rnode{\mathsf{O}}}
\newcommand{\rnulltree}{\rnode{\mathsf{O}}}
\newcommand{\rel}[2]{{\rnode{#1}}{{}_{_{#2}}}}
\newcommand{\rextn}[1]{\rel{#1}{\set{E}}}
%\newcommand{\rchild}[1]{\rel{#1}{\msdec{\set{C}}}}
\newcommand{\rchild}[1]{\rel{#1}{\set{C}}}
\newcommand{\rlabel}[1]{\rel{#1}{\mathsf{P}}}
\newcommand{\rv}[1]{\rel{#1}{\nu}}
\newcommand{\rinertia}[1]{\rel{#1}{r}}
\newcommand{\rK}[1]{\rel{#1}{K}}
\newcommand{\rI}[1]{\rel{#1}{I}}
\newcommand{\rM}[1]{\rel{#1}{M}}

\newcommand{\trnode}[1]{\(\rnode{#1}\)}
\newcommand{\tronetree}{\(\rnode{\I}\)}
\newcommand{\trzerotree}{\(\rnode{\mathsf{O}}\)}
\newcommand{\trnulltree}{\(\rnode{\mathsf{O}}\)}
\newcommand{\trel}[2]{\({\rnode{#1}}_{#2}\)}
\newcommand{\trextn}[1]{\(\rel{#1}{\set{E}}\)}
\newcommand{\trchild}[1]{\(\rchild{#1}\)}
\newcommand{\trlabel}[1]{\(\rel{#1}{\mathsf{P}}\)}
\newcommand{\trv}[1]{\(\rel{#1}{\nu}\)}
\newcommand{\trsubj}[1]{\(\rel{#1}{\delta}\)}
\newcommand{\trinertia}[1]{\(\rel{#1}{r}\)}
\newcommand{\trK}[1]{\(\rel{#1}{K}\)}
\newcommand{\trI}[1]{\(\rel{#1}{I}\)}
\newcommand{\trM}[1]{\(\rel{#1}{M}\)}

\newcommand{\rlabels}[1] {\mb{L}(\rnode{#1})}
\newcommand{\notrlabels}[1] {\overline{\mb{L}}(\rnode{#1})}

\newcommand{\srestrictedtor}[2]{\set{#1}\vert_{\rlabels{#2}}}
\newcommand{\nsrestrictedtor}[2]{\set{#1}\vert_{\notrlabels{#2}}}
\newcommand{\restrictedtor}[2]{\rnode{#1}\vert_{\rlabels{#2}}}
\newcommand{\nrestrictedtor}[2]{\rnode{#1}\vert_{\notrlabels{#2}}}

\newcommand{\childOr}[5]{\bigl(\nrestrictedtor{#1}{#2}\cup\nrestrictedtor{#2}{#1}\cup\{\rnode{#3}{#5}\rnode{#4}:\rnode{#3}\in\restrictedtor{#1}{#2}\mand\rnode{#4}\in\restrictedtor{#2}{#1}\mand\qlabel{#3}\sim\qlabel{#4}\}\bigr)\setminus\{\zerotree\}}
\newcommand{\childObr}[5]{\bigl(\nrestrictedtor{#1}{#2}\cup\nrestrictedtor{#2}{#1}\\
&\qquad\cup\{\rnode{#3}{#5}\rnode{#4}:\rnode{#3}\in\restrictedtor{#1}{#2}\mand\rnode{#4}\in\restrictedtor{#2}{#1}\mand\qlabel{r}\sim\qlabel{s}\}\bigr)\setminus\{\zerotree\}}

\newcommand{\childIr}[5]{\{\rnode{#3}\,{#5}\,\rnode{#4}:\rnode{#3}\in\restrictedtor{#1}{#2}\mand\rnode{#4}\in\restrictedtor{#2}{#1}\mand\qlabel{#3}=\qlabel{#4}\}\setminus\nullspace}

\newcommand{\extnOr}[5]{\bigl(\nrestrictedtor{#1}{#2}\cup\nrestrictedtor{#2}{#1}\cup\{\rnode{#3}{#5}\rnode{#4}:\rnode{#3}\in\restrictedtor{#1}{#2}\mand\rnode{#4}\in\restrictedtor{#2}{#1}\mand\qlabel{#3}=\qlabel{#4}\}\bigr)\setminus\{\zerotree\}}
\newcommand{\extnObr}[5]{\bigl(\nrestrictedtor{#1}{#2}\cup\nrestrictedtor{#2}{#1}\\
&\qquad\cup\{\rnode{#3}{#5}\rnode{#4}:\rnode{#3}\in\restrictedtor{#1}{#2}\mand\rnode{#4}\in\restrictedtor{#2}{#1}\mand\qlabel{r}=\qlabel{s}\}\bigr)\setminus\{\zerotree\}}

\newcommand{\extnIr}[5]{\{\rnode{#3}\,{#5}\,\rnode{#4}:\rnode{#3}\in\restrictedtor{#1}{#2}\mand\rnode{#4}\in\restrictedtor{#2}{#1}\mand\qlabel{#3}=\qlabel{#4}\}\setminus\nullspace}

\newcommand{\expandednorm}[5]{\frac{#1}{\card{#2}^#3}\sum_{\node{#4}\in{#5}}\nabs{\node{#4}}}
\newcommand{\quadexpandednorm}[5]{\frac{#1}{\card{#2}^#3}\quad\sum_{\node{#4}\in{#5}}\nabs{\node{#4}}}

\newcommand{\stexpandednorm}[5]{\frac{#1}{\card{#2}^#3}\quad\stsum{\node{#4}\in{#5}}\nabs{\node{#4}}}
\newcommand{\quadstexpandednorm}[5]{\frac{#1}{\card{#2}^#3}\quad\stsum{\quad\node{#4}\in{#5}}\quad\nabs{\node{#4}}}
\newcommand{\qquadstexpandednorm}[5]{\frac{#1}{\card{#2}^#3}\qquad\stsum{\qquad\node{#4}\in{#5}}\qquad\nabs{\node{#4}}}

\newcommand{\mstexpandednorm}[5]{\frac{#1}{\card{#2}^#3}\quad\mstsum{\node{#4}\in{#5}}\nabs{\node{#4}}}

%--

% Must be math mode!
\newcommand{\half}[0]{\frac{1}{2}}

%?\newcommand{\rrangle} {\rangle\hspace{-2.5pt}\rangle}
%?\newcommand{\llangle} {\langle\hspace{-2.5pt}\langle}

\newcommand{\lHash}{\mbox{\ooalign{\(=\)\cr\hidewidth\(\|\)\hidewidth\cr}}}
\newcommand{\rHash}{\mbox{\ooalign{\(=\)\cr\hidewidth\(\|\)\hidewidth\cr}}}

\newcommand{\matr}[1]{\ensuremath{\mathsfsl{#1}}}

%% %% subscript lower on the \vert and smaller, you see
%\newcommand{\norm}[1]{\lVert{#1}\rVert}
\newcommand{\nnorm}[1]{\llangle{#1}\rrangle}
\newcommand{\prennorm}[1]{\langlebar{#1}\ranglebar}
\newcommand{\norm}[1]{\langle{#1}\rangle}
\newcommand{\card}[1]{\lVert{#1}\rVert}
\newcommand{\abs}[1]{\lvert{#1}\rvert}
\newcommand{\lpar}{\llparenthesis}
\newcommand{\rpar}{\rrparenthesis}
\newcommand{\nabs}[1]{\lpar{#1}\rpar}
%\newcommand{\nabs}[1]{\norm{#1}}
\newcommand{\magn}[1]{\nabs{#1}}
\newcommand{\fmag}[1]{\nabs{#1}}
\newcommand{\Tcard}[1]{\card{#1}_{\intercal}}
\newcommand{\content}[1]{\llbracket{#1}\rrbracket}

%\newcommand{\nabs}[1]{\mb{\lvert}{#1}\mb{\rvert}} 
%\newcommand{\nabs}[1]{\mb{\lvert}{#1}\mb{\rvert}_{{}_\Sigma}} 
%\newcommand{\nabs}[1]{\lVert{#1}\rVert}
%\newcommand{\nabs}[1]{\abs{#1}}
%\newcommand{\fmag}[1]{\lvert\lvert{#1}\rvert\rvert}
%\newcommand{\fmag}[1]{\overset{\nabs{#1}}{
%\newcommand{\fmag}[1]{\norm{#1}}

\newcommand{\NodeR}[2]{\left({#1},{#2}\right)}
\newcommand{\NNodeR}[2]{\bigl({#1},{#2}\bigr)}
\newcommand{\NNNodeR}[2]{\Bigl({#1},{#2}\Bigr)}
\newcommand{\NNNNodeR}[2]{\BIGL({#1},{#2}\BIGR)}

\newcommand{\Node}[3]{\left({#1},{#2}, {#3}\right)}
\newcommand{\NNode}[3]{\bigl({#1},{#2}, {#3}\bigr)}
\newcommand{\NNNode}[3]{\Bigl({#1},{#2}, {#3}\Bigr)}
\newcommand{\NNNNode}[3]{\BIGL({#1},{#2}, {#3}\BIGR)}

\newcommand{\SNode}[1]{\Node{\rv{#1}}{\rlabel{#1}}{\rchild{#1}}}
\newcommand{\SNNode}[1]{\NNode{\rv{#1}}{\rlabel{#1}}{\rchild{#1}}}
\newcommand{\SNNNode}[1]{\NNNode{\rv{#1}}{\rlabel{#1}}{\rchild{#1}}}
\newcommand{\SNNNNode}[1]{\NNNNNode{\rv{#1}}{\rlabel{#1}}{\rchild{#1}}}

%\newcommand{\SNode}[3]{({#1}, {#2}, {#3})}
%\newcommand{\SNNode}[3]{\bigl({#1}, {#2}, {#3}\bigr))}
%\newcommand{\SNNNode}[3]{\Bigl({#1}, {#2}, {#3}\Bigr))}
%\newcommand{\SNNNNode}[3]{\BIGL({#1}, {#2}, {#3}\BIGR))}

%\newcommand{\NNode}[3]{\bigl({#1}, {#2}, {#3}\bigr))}
%\newcommand{\NNNode}[3]{\Bigl({#1}, {#2}, {#3}\Bigr))}
%\newcommand{\NNNNode}[3]{\BIGL({#1}, {#2}, {#3}\BIGR))}

\newcommand{\NodeIII}[3]{\left({#1},{#2}, {#3}\right)}
\newcommand{\NNodeIII}[3]{\bigl({#1},{#2}, {#3}\bigr)}
\newcommand{\NNNodeIII}[3]{\Bigl({#1},{#2}, {#3}\Bigr)}
\newcommand{\NNNNodeIII}[3]{\BIGL({#1},{#2}, {#3}\BIGR)}

\newcommand{\SNodeIII}[1]{\NodeIII{\rv{#1}}{\rlabel{#1}}{\rchild{#1}}}
\newcommand{\SNNodeIII}[1]{\NNodeIII{\rv{#1}}{\rlabel{#1}}{\rchild{#1}}}
\newcommand{\SNNNodeIII}[1]{\NNNodeIII{\rv{#1}}{\rlabel{#1}}{\rchild{#1}}}
\newcommand{\SNNNNodeIII}[1]{\NNNNNodeIII{\rv{#1}}{\rlabel{#1}}{\rchild{#1}}}

%\newcommand{\SNodeIII}[3]{({#1}, {#2}, {#3})}
%\newcommand{\SNNodeIII}[3]{\bigl({#1}, {#2}, {#3}\bigr))}
%\newcommand{\SNNNodeIII}[3]{\Bigl({#1}, {#2}, {#3}\Bigr))}
%\newcommand{\SNNNNodeIII}[3]{\BIGL({#1}, {#2}, {#3}\BIGR))}

%\newcommand{\NNodeIII}[3]{\bigl({#1}, {#2}, {#3}\bigr))}
%\newcommand{\NNNodeIII}[3]{\Bigl({#1}, {#2}, {#3}\Bigr))}
%\newcommand{\NNNNodeIII}[3]{\BIGL({#1}, {#2}, {#3}\BIGR))}

\newcommand{\NodeII}[2]{\left({#1},{#2}\right)}
\newcommand{\NNodeII}[2]{\bigl({#1},{#2}\bigr)}
\newcommand{\NNNodeII}[2]{\Bigl({#1},{#2}\Bigr)}
\newcommand{\NNNNodeII}[2]{\BIGL({#1},{#2}\BIGR)}

\newcommand{\SNodeII}[1]{\NodeII{\rv{#1}}{\rchild{#1}}}
\newcommand{\SNNodeII}[1]{\NNodeII{\rv{#1}}{\rchild{#1}}}
\newcommand{\SNNNodeII}[1]{\NNNodeII{\rv{#1}}{\rchild{#1}}}
\newcommand{\SNNNNodeII}[1]{\NNNNNodeII{\rv{#1}}{\rchild{#1}}}

%\newcommand{\SNodeII}[2]{({#1}, {#2})}
%\newcommand{\SNNodeII}[2]{\bigl({#1}, {#2}\bigr))}
%\newcommand{\SNNNodeII}[2]{\Bigl({#1}, {#2}\Bigr))}
%\newcommand{\SNNNNodeII}[2]{\BIGL({#1}, {#2}\BIGR))}

%\newcommand{\NNodeII}[2]{\bigl({#1}, {#2}\bigr))}
%\newcommand{\NNNodeII}[2]{\Bigl({#1}, {#2}\Bigr))}
%\newcommand{\NNNNodeII}[2]{\BIGL({#1}, {#2}\BIGR))}


%\newcommand{\Tcard}[1]{\norm{#1}}
%\newcommand{\Tcard}[1]{\lHash #1 \rHash}

%%\newcommand{\capcross}{\mbox{\ooalign{\(\cap\)\cr\hidewidth\(\times\)\hidewidth\cr}}}
%%\newcommand{\cupcross}{\mbox{\ooalign{\(\cup\)\cr\hidewidth\(\times\)\hidewidth\cr}}}

%%\newcommand{\capplus}{\mbox{\ooalign{{\large\(\cap\)}\cr\hidewidth\({\text{\tiny{+}}}\)\hidewidth\cr}}}
%%\newcommand{\cupplus}{\mbox{\ooalign{{\large\(\cup\)}\cr\hidewidth\({\text{\tiny{+}}}\)\hidewidth\cr}}}

%%\newcommand{\capdot}{\dot{\cap}}
%%\newcommand{\cupdot}{\dot{\cup}}

%%\newcommand{\symint}{\capcross}
%%\newcommand{\symunion}{\cupcross}

% \makeatletter
%\def\moverlay{\mathpalette\mov@rlay}
%\def\mov@rlay#1#2{\leavevmode\vtop{%
%    \baselineskip\z@skip \lineskiplimit-\maxdimen
%    \ialign{\hfil$\m@th#1##$\hfil\cr#2\crcr}}}
%\newcommand{\charfusion}[3][\mathord]{
%     #1{\ifx#1\mathop\vphantom{#2}\fi
%         \mathpalette\mov@rlay{#2\cr#3}
%       }
%     \ifx#1\mathop\expandafter\displaylimits\fi}
% \makeatother

%\newcommand{\cupplus}{\charfusion[\mathbin]{\cup}{+}}
%\newcommand{\bigcupplus}{\charfusion[\mathop]{\bigcup}{+}}

\newcommand{\lbl}[1]{\mathsf{#1}}
\newcommand{\Tlbl}[1]{$\mathsf{#1}$}

%% Declare the sdiff operator
\newcommand{\sdiff}[4]{#1(#3,#4)#2(#4) - #1(#4,#3)#2(#3)}
\newcommand{\sigmoid}[1]{\frac{e^{#1}}{1+e^{#1}} }

% Appendix things ...
\newcommand{\tqset}[1]{\(\RNG{\mb{#1}}\)}


%%%%%%%%%%%%%%%%%%%%%%%%%%%%%%%%%%%%%%%%%% local things %%%%%%%%%%%%%%%%%%%%%%%%%%%%%%%%%%%%%%%%%%

%% ODD sections
\newcommand{\ODD}[0]{\section{Overview: an ODD model description}}
\newcommand{\oddPurpose}[0]{\subsection{Purpose}}
\newcommand{\oddEntitiesEtc}[0]{\subsection{Entities, state variables and scales}}
\newcommand{\oddEntities}[0]{\subsubsection{Entities}}
\newcommand{\oddStateVars}[0]{\subsubsection{State variables}}
\newcommand{\oddScales}[0]{\subsubsection{Scales}}

\newcommand{\oddProcessOverviewAndScheduling}[0]{\subsection{Process overview and scheduling}}

\newcommand{\oddDesign}[0]{\subsection{Design concepts}}
\newcommand{\oddDesignEmergence}[0]{\subsubsection{Emergent features}}
\newcommand{\oddDesignAdaptation}[0]{\subsubsection{Adaptation}}
\newcommand{\oddDesignObjectives}[0]{\subsubsection{Objectives}}
\newcommand{\oddDesignLearning}[0]{\subsubsection{Learning}}
\newcommand{\oddDesignPrediction}[0]{\subsubsection{Prediction}}
\newcommand{\oddDesignSensing}[0]{\subsubsection{Sensing}}
\newcommand{\oddDesignInteraction}[0]{\subsubsection{Interaction}}
\newcommand{\oddDesignStochasticity}[0]{\subsubsection{Stochasticity}}
\newcommand{\oddDesignCollectives}[0]{\subsubsection{Collectives}}
\newcommand{\oddDesignObservation}[0]{\subsubsection{Observation}}
\newcommand{\oddDesignExplanation}[0]{\subsubsection{Explanation}}

\newcommand{\oddInitialization}[0]{\subsection{Initialization}}
\newcommand{\oddInputData}[0]{\subsection{Input Data}}
\newcommand{\oddSubmodels}[0]{\subsection{Submodels}}

\newcommand{\fruit}[0]{\emph{fruit}}
\newcommand{\seed}[0]{\emph{seed}}
\newcommand{\seeds}[0]{\emph{seeds}}
\newcommand{\plant}[0]{\emph{plant}}
\newcommand{\plants}[0]{\emph{plants}}
\newcommand{\herbivore}[0]{\emph{herbivore}}
\newcommand{\herbivores}[0]{\emph{herbivores}}
\newcommand{\carnivore}[0]{\emph{carnivore}}
\newcommand{\carnivores}[0]{\emph{carnivores}}

\newcommand{\PN}[1]{\hat{N_{\mt{#1}}}}
\newcommand{\N}[1]{N_{\mt{#1}}}
\newcommand{\R}[1]{\rho_{\mt{#1}}}
\newcommand{\F}[1]{f_{\mt{#1}}}
\newcommand{\Gm}[1]{\Gamma_{\mt{#1}}}
\newcommand{\E}[1]{E_{\mt{#1}}}
\newcommand{\K}[1]{K_{\mt{#1}}}
\newcommand{\mort}[1]{\Omega_{\mt{#1}}}
\newcommand{\imort}[1]{\Omega_{\text{ind}\mt{#1}}}
\newcommand{\starve}[2]{\min\biggl(0,\frac{\N{#2}-\omega_{\mt{#1 #2}}(x)\N{#1}}{\N{#2}}\biggr)}


%-- Flags, such as line numbers


%-- Environments
\newenvironment{indented}{\begin{adjustwidth}{24pt}{}}{\end{adjustwidth}}

%-- Commands for fonts

%-- Newcommands

%% declares the character degreesC (U+2103) to map to the \textcelsius
%%function \DeclareUnicodeCharacter{"2103}{\textcelsius}

\newcommand{\resetfonts}[0]{\fontencoding{\encodingdefault}\fontfamily{\familydefault}\fontseries{\seriesdefault}\fontshape{\shapedefault}\selectfont}

%%\newcommand{\SupData}\newcommand{\appB}

\newcommand{\Mtt}[1]{\mathtt{#1}}\newcommand{\ttt}[1]{\texttt{#1}}

%\newcommand{\SD}[0]{\textsf{\emph{SD}}}
%\newcommand{\IB}[0]{\textsf{\emph{IB}}}

\newcommand{\SD}[0]{\emph{SD}}\newcommand{\IB}[0]{\emph{IB}}

\newcommand{\location}[0]{\mathit{Locus}}
\newcommand{\mass}[0]{\mathit{Mass}}
\newcommand{\foragecount}[0]{\mathit{ForageCt}}
\newcommand{\peakmass}[0]{\mathit{PkMass}}
\newcommand{\hungertime}[0]{\mathit{Hungry}}
\newcommand{\satedtime}[0]{\mathit{Sated}}
\newcommand{\growth}[0]{\mathit{Growth}}
\newcommand{\growthnstarve}[0]{\mathit{Growth\&Starv}}
\newcommand{\germination}[0]{\mathit{Germ}}

\newcommand{\reproduction}[0]{\mathit{Repr}}
\newcommand{\predationmort}[0]{\mathit{PredMort}}
\newcommand{\naturalmort}[0]{\mathit{NatMort}}

\newcommand{\mature}[0]{\mathit{Mature}}
\newcommand{\fruits}[0]{\mathit{Fruits}}
\newcommand{\preylist}[0]{\mathit{PreyList}}

\newcommand{\mathsc}[1]{\text{\textsc{#1}}}
\newcommand{\Urnd}[0]{\ensuremath{\mathrm{rnd}_{_{0,1}}}}
%%\def\Urnd{{\mathrm{Urnd}_{_{{{(0,1)}}}}}}
\newcommand{\lcount}[1]{\mathrm{len(#1)}}
\newcommand{\AddFruit}[2]{\mathsc{AddFruit}(#1 #2)}
\newcommand{\Eat}[3]{\mathsc{Eat}(#1, #2, #3)}
\newcommand{\Growth}[3]{\mathsc{Growth}_{\Mtt{#1}}(#2, #3)}


\newcommand{\LocallyCrowded}[1]{\mathsc{Crowded}_{\Mtt{#1}}}
\newcommand{\Updatetree}[1]{\mathsc{UpdateStateTrees}_{\Mtt{#1}}}
\newcommand{\Die}[0]{\textsc{Die}}
\newcommand{\PreyPresent}[3]{\mathsc{PreyPresent}_{\Mtt{#1}}(#2,#3)}
\newcommand{\Reproduce}[2]{\mathsc{Reproduce}_{\Mtt{#1}}(#2)}
\newcommand{\Migrate}[2]{\mathsc{Migrate}_{\Mtt{#1}}(#2)}

\newcommand{\SuppMaterial}[0]{\textsl{Supplementary Material}}

%\newcommand{\stree}[3]{\ensuremath{(\!({\mc{#1}}, #2, \{#3\})\!)}}
%\newcommand{\stree}[3]{\ensuremath{(\!:\mc{#1}, #2, \{#3\}:\!)}}
%\newcommand{\stree}[3]{\ensuremath{(\!\vert\mc{#1}, #2,
%\{#3\}\vert\!)}} \newcommand{\stree}[3]{\ensuremath{(\!\vert\mc{#1},
%#2, \{#3\}\vert\!)}} \newcommand{\etree}[3]{\ensuremath{(\mc{#1},
%#2, \{#3\})}}

%\newcommand{\stree}[3]{\ensuremath{(\!\vert\mc{#1}, #2,
%\{#3\}\vert\!)}} \newcommand{\etree}[3]{\ensuremath{(\mc{#1}, #2,
%\{#3\})}}

%\newcommand{\sxtree}[1]{\mathpzc{root}, #1, }
%\newcommand{\nxtree}[2]{\mathpzc{#1}, #2, }

%%\newcommand{\tree}[0]{\textsf{tree}}
%%\newcommand{\trees}[0]{\textsf{trees}}
%%\newcommand{\Tree}[0]{\textsf{Tree}}
%%\newcommand{\Trees}[0]{\textsf{Trees}}

%%\newcommand{\stree}[0]{\textsf{status tree}}
%%\newcommand{\strees}[0]{\textsf{status trees}}
%%\newcommand{\Stree}[0]{\textsf{Status tree}}
%%\newcommand{\Strees}[0]{\textsf{Status trees}}

\DeclareMathOperator{\statevector}{state}
\DeclareMathOperator{\score}{score}
\DeclareMathOperator{\assessa}{assess_1}
\DeclareMathOperator{\assess}{assess}
\DeclareMathOperator{\Rassess}{rep\_assess}
\DeclareMathOperator{\needs}{needs}

\newcommand{\polystruct}[0]{ring}
\newcommand{\polyrat}[0]{}
\newcommand{\polyform}[0]{multinomial}
\newcommand{\polyforms}[0]{multinomials}
\newcommand{\polytype}[0]{\polystruct\ of \polyrat\ \polyform}
\newcommand{\polytypes}[0]{\polystruct\ of \polyrat\ \polyforms}



%\newcommand{\UpnotVp}[2]{{\set{#1}{\lnot\set{#2}}}}
%\newcommand{\UpandVp}[2]{{\set{#1}{\wedge\set{#2}}}}

%\newcommand{\polystruct}{ring\ }
%\newcommand{\polyrat}{\ }
%\newcommand{\polytype}{commutative rng of\ }
\typeout{manifest.tex finished ******* }

% \message{Finished Manifest =========================================}

%- Environments

\newenvironment{indented}{\begin{adjustwidth}{24pt}{}}{\end{adjustwidth}}

%% declares the character degreesC (U+2103) to map to the \textcelsius function
%% \DeclareUnicodeCharacter{"2103}{\textcelsius}

% Newcommands
\newcommand{\mt}[1]{\mathtt{#1}}
\newcommand{\mB}[1]{\mathbb{#1}}
\newcommand{\mb}[1]{\mathbf{#1}}
\newcommand{\mf}[1]{\mathfrak{#1}}
\newcommand{\mc}[1]{\mathpzc{#1}}
% \newcommand{\mc}[1]{\mathcal{#1}}

\newcommand{\filename}[1]{\texttt{#1}}
\newcommand{\statevariable}[1]{\textit{#1}}
\newcommand{\rclass}[1]{\textsf{<#1>}}
\newcommand{\method}[1]{\textbf{\textit{#1}}}

\newcommand{\landscape}[0]{\rclass{landscape}}
\newcommand{\environment}[0]{\rclass{environment}}
\newcommand{\ecoservice}[0]{\rclass{ecoservice}}
\newcommand{\patch}[0]{\rclass{patch}}
\newcommand{\patches}[0]{{\patch}es}
\newcommand{\dynamicpatch}[0]{\rclass{dynamic-patch}}
\newcommand{\dynamicpatches}[0]{{\dynamicpatch}es}
\newcommand{\diffeqsystem}[0]{\rclass{diffeq-system}}
\newcommand{\populationsystem}[0]{\rclass{population-system}}
\newcommand{\polygonshape}[0]{\rclass{polygonshape}}
\newcommand{\circleshape}[0]{\rclass{circleshape}}

\newcommand{\extset}[0]{\method{extset!}
\newcommand{\extget}[0]{\method{extget}


\begin{document}
\title[Remodel]{}

% Some useful definitions...


\section{Discussion of the model structure}

\subsection{Landscape classes}
The classes which represent the ``landscape'' of the model are
ultimately related to the \environment class described in
\filename{remode-classes.scm}. These classes would generally be
responsible for representing physical componenets, such as the
terrain, groundwater and surface water, but they may also be used to
represent elements that might be referred to with mass-nouns, such as
grass, forest or foliage.  

This class maintains the physical boundaries of the environment
(minimum and maximum values for spatial ordinates, a default spot
height, an inner radius and an outer radius which can be used to speed
``presence'' checks, and a ``representation'' which isi usually
something like a polygon, a DEM or some similar object.

The \landscape code consists mainly of a few classes.  Strictly
speaking, a \landscape can include the terrain, and various
\textit{attributes} which are characterised by numbers which can
increase or decrease. The \textit{attributes} are usually encapsulated
in \patches which are spatially anchored regions which (properly)
should not overlap significantly.  There is ready support for
rectangular and hexagonal regions that cover the the domain of the
\landscape in a tesselation, though there are also circular
representations which may be appropriate in simulations where complete
coverage is not required -- these patch shapes obviously don't
tesselate!  A terrain function is also supplied when creating a
landscape. This function typically expects a locus and returns an
altitude relative to the datum  the model.

The attributes mentioned above are implemented as members of the
\ecoservice class.  This is a class which basically maintains some
``value'' associated with the patch, such as the biomass of grass, or
the number of pademelons. It is able to act either as an agent
maintaining the state of some environmental component or as a proxy
for a set of other agents which should be treated as a group (perhaps
a population of rabbits) using \extset and \extget calls. The
water-table in a patch may be modelled using an ecoservice: its
internal recharge model will cause the amount of available water to
increase as a function of its natural recharge rate, while flora in
the patch and local industry which extracts water will reduce the
available water with appropriate calls to secure the water they need.
Ecoservices keep track of the patches for which they are relevant and
\textit{vis-a-versa}. Externally modelled components that are treated
as ecoservices would be polled at the beginning of the ecoservice's
slice of time in order to set the numeric \statevariable{value} slot
in the \ecoservice agent, and at the end of the update step the
corresponding values for the agents polled would be adjusted
appropriately. 

\patches maintain a list of their ecoservices, their spatial location
and footprint, an optional ``caretaker'' function which is called at
each timestep, a variable (\textit{notepad} containing a list which
can be used for any cross-tick storage, and a list of the patch's
neighbours. Neigbours in the list are represented by lists of the form
'\textit{(neighbour-patchagent number | symbol \#!optional extras)}';
the \textit{extras} component is not required and currently not used
in any of the submodels.  The numeric second argument represents the
``length'' of the boundary between the agent and its neighbour (which
can be used to mediate the exchange between them).

\dynamicpatches are derived from both \patch and \diffeqsystem and
allows the patch to behave as an ensemble which is governed by a
system of differential equations. Interactions between neigbouring
sources and sinks for the ecoservices represented in the system are
supported and worked into the dynamics of a timestep.

There are also simple geometric classes that correspond to ``regions''
such as \polygonshape and \circleshape; while squares and hexagons
feature as tiles in a landscape, they are simply instances of
\polygonshape.

\subsubsection{Constructing landscapes}

The declaration

\begin{verbatim}
  (define HeronIs 
     (make <landscape>
        'mname "Heron Island" 
        'default-value (* 0.5 meters)
        heron-dem-func

        









\end{document}
