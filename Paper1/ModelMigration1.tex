\documentclass{article}
\usepackage{geometry,amsmath,enumerate,color}
\geometry{a4paper}

%%%%%%%%%% Start TeXmacs macros
\newcommand{\tmem}[1]{{\em #1\/}}
\newcommand{\tmmathbf}[1]{\ensuremath{\boldsymbol{#1}}}
\newcommand{\tmstrong}[1]{\textbf{#1}}
\newenvironment{enumerateroman}{\begin{enumerate}[i.] }{\end{enumerate}}
\newenvironment{itemizedot}{\begin{itemize} \renewcommand{\labelitemi}{$\bullet$}\renewcommand{\labelitemii}{$\bullet$}\renewcommand{\labelitemiii}{$\bullet$}\renewcommand{\labelitemiv}{$\bullet$}}{\end{itemize}}
\definecolor{grey}{rgb}{0.75,0.75,0.75}
\definecolor{orange}{rgb}{1.0,0.5,0.5}
\definecolor{brown}{rgb}{0.5,0.25,0.0}
\definecolor{pink}{rgb}{1.0,0.5,0.5}
%%%%%%%%%% End TeXmacs macros

\begin{document}



\title{Increasing model efficiency by dynamically changing model
representations}\author{Randall Gray\\
CSIRO Division of Marine and Atmospheric Research\andSimon Wotherspoon\\
University of Tasmania}
\begin{abstract}
\begin{verse}
We use one model, \\
but many may be more deft. \\
Is it worth trying?
\end{verse}

There are a number of strategies to dealing with modelling large
complex systems -- in our case, large marine ecosystems. These systems
are often comprised of many submodels, each representing a particular
process or participant in a way which tries to capture the dynamics
which contribute to the trajectory of the system.  The balance between
the acceptible modelling error and the run-time often dictates the
form of submodels.  There may be scope to improve the position of this
balance point in both regards by structuring models so that submodels
may change their algorithmic representation and state space in
response to their local state and the state of the model as a whole.  
\\
This paper uses an example system consisting of a single population of
animals which periodically encounter a diffuse contaminant in a
localised region to compare the performance of a population-based
representation, an individual-based representation and a model which
allows the representation to change from one to the other based on the
likelihood of contaminant contact.  The resulting run-times and
contaminant dynamics associated with each of the stategies suggest
that such a mutating--model approach may be an avenue to gain accuracy
within time constraints, time within accuracy constraints, or both.
\end{abstract}


\maketitle

\section{Introduction}

Models of the effects of human interactions with the environment and with
animal and plant populations are often are faced with the dilemma of choosing
an appropriate scale at which to model the interactions. Population level
effects are likely to be modelled with an analytic approach, but an
agent-based or individual-based approach may be more appropriate if our system
exhibits behaviour which isn't compatible with mean-field assumptions.

There is a body of literature stretching back several decades which discusses
individual-based modelling as an effective approach to modelling systems where
individual variability is perceived to be an important driver of the system's
dynamics, notably JABOWA and its derivatives (refi://Botkin72:1,
refi://Botkin72:2). Since the early 1980s, a rapidly increasing number of
significant papers and books have appeared which address the use of
individual-based models across a broad \ and with discussion of the relative
strengths and weaknesses of the approach (such as refi://Huston88:1,
refi://DeAngelis92:1 and refi://Grimm05:1). The use of classical (equation
based) models to explore populations and ecological systems goes back much
further, arguably to 1798 with Malthus's {\tmem{{\tmstrong{{\tmstrong{An Essay
on the Principle of Population}}}}}}.

In some sense these approaches represent the extrema in a spectrum which
exhibits, through its range, varying degrees of aggregation and resolution in
time, space, and membership. Models at one end or the other are obvious
representatives of a ``choice of paradigm,'' but there are a large number of
models which incorporate representatives from both extrema, and indeed adopt
intermediate (super-individual) representations, such as described by
refi://Scheffer95:1.

As ecosystem models become broader in scope, including more species and richer
environments, the notion that a model can be identified with some particular
region of this spectrum starts break down. Simulation models often embed the
subject of study within an ``environment'' made up of primary data and other
processes within the system. These data and processes which comprise the
environment may simultaneously lie in quite a number of places between purely
equation-based representations and individual-based representations. The
actual implementation may be anything from a set of quite distinct submodels
which are coupled together but retain their independence and, in some sense,
stand as models in their own right, to a corpus of code where the submodels
are seamlessly integrated and the aggregated model is presented, like an
organism, as an indivisible entity.

There are quite a number of well established ways of integrating submodels,
many of which achieve efficiency, either in run-time or accuracy, by running
different components of the model at different temporal or spatial scales such
as with variable-speed splitting and model nesting. Similarly, many of the
numeric techniques we routinely apply in our models use adaptive steps in
their solution (e.g. Runge-Kutta and root finding).

As the climate changes and the environment we live in and use drifts further
from its familiar state, there is a corresponding need to manage human
interaction with ecosystems more carefully. The scope of ecosystem models is
steadily increasing (refi://DeAngelis98:1, refi://Harvey03:1
refi://Fulton04:1, refi://Gray06:1){\color{red} }. Models are including more
functional groups within their modeled ecosystems and the interactions between
components are becoming more detailed. Addressing the increased demand for
detail is costly in terms of computational load: individual-based models of
populations may be very good at capturing the vulnerability to exceptional
events, but such simulations take a long time. Worse still, much of this time
may be spent with the model in a largely unchallenging or uninteresting part
of its state-space.

As a simple example, we might consider the motion of a mass capable of
acceleration; perhaps a rocket or a taxi. The rules of Newtonian motion make a
good model as long as the mass doesn't move too quickly, and, while
relativistic motion is much more accurate at high speeds, it is more expensive
to calculate. A simulation of a spacecraft (or taxi) that approaches the speed
of light would spend a lot of time calculating relativistic motion when a
simpler model would be adequate. It is easy to imagine that we might use
Newton's equations when the velocities are low (most of the state space), but
shift to relativistic motion when we stray from the areas where Newton is
comfortably accurate.

The notion explored in this paper is that we might change the
{\tmem{representation}} of a submodel based on its location in its
state-space, and that by doing so we may actually be able to simulate the
system more effectively. To some degree modellers do this anyway: time-steps
or spatial resolutions are changed, particular code paths may be by-passed
according to local conditions within the simulation, or additional
calculations might be performed to reduce the error when the state is changing
rapidly; but these optimisations are largely optimisations of the
{\tmem{encoding}} of the model or submodels, rather than an actual change in
representation. In contrast, moving from a population modelled by a set of
differential equations to a model tracking the life-history of a large number
of individuals involves a change of domain, state-space and changed
assumptions concerning the homogeneity of the consituent members of the
population being modelled.

This approach to the problem of managing complex simulations has been
developed in the light of experience with several large scale human-ecosystem
interaction models (refi://Lyne94:1, refi://Gray06:1, and now the modelling of
the interactions in a study of the region containing the Ningaloo Marine
Park). In each of these studies a significant component has been to simulate
the interaction between organisms and contaminant plumes. The focus in Lyne
{\tmem{et al.}} was the potential for the percolation of contaminants
originating in industrial waste up through the food chain into commercially
exploited fish stocks. Gray {\tmem{et al.}} developed a regional model
focussed on assessing ways of managing the impact of human activity on the
biological systems along the Northwest Shelf of Australia, one component of
which was the effects of flushing the bitterns from salt production into the
surrounding environment.

refi://Monte09:1 presents a lucid discussion of {\tmem{contaminant
migration-population effects}} models. These models incorporate the movement
of populations and their internal distribution, the transport of contaminants
through the system via biotic and abiotic pathways, and the changes in
behaviour and population dynamics associated with contamination. Monte
discusses \ a method of coupling the equations which govern contaminant
dispersion and the equations for population dynamics and migration. The
technique depends on particular conditions on the nature of some of the
equations. He summarises the implications of the conditions, stating that the
class of systems where the ``movement of animals, the death and birthrates of
individuals in $\tmmathbf{x}$ at instant $t$ depend on previously occupied
positions'' is not generally amenable to the approach, and suggests that
repeated simulations of many individuals is an appropriate way of dealing with
this situation.

The treatment of contaminants, in particular, is expensive in terms of
run-time and memory use -- in the model described by refi://Gray06:1
contaminant modelling increased the time taken by roughly an order of
magnitude. In all of these studies a considerable amount of time was spent in
regions where no interaction with contaminant plumes was possible.

Running a complex model and maintaining its state in a region where a simple
model may perform better imposes a burden which is probably unnecessary. There
is potential for both significant improvements in run-time and the reduction
of error. There are four basic questions that arise: ``When should a model
change representation?'', ``What data needs to persist across
representations?'', ``How is the initial state for a new representation
constructed?'' and finally, ``How should the error associated with the loss of
state information be managed?'' In general, the answer to these questions is
specific to the ensemble of submodels in question. Before expending the
resources and effort required to implement such a strategy in a large scale
model there needs to be, {\tmem{prima facie}}, a demonstration that the notion
is worth pursuing. The aim of this paper is to provide this demonstration
rather than to develop a body of techniques supporting the approach.

We present a model of organisms moving along a simple migratory path which
intersects a region with a field of contamination with varying levels. This
model exhibits fundamental attributes of larger studies of pollutant/ecosystem
interactions (refi://Lyne94:1 and refi://Gray06:1), and while it isn't
intended to accurately represent any particular system, it might loosely
correspond to some body of water influenced by contaminant loads associated
with terrestrial runoff resulting from intense rainfalls. Its role is to help
explore some of the issues associated with changing representations and to
test the hypothesis that we might gain efficiency in simulation time,
representational accuracy, or both.

Our model uses basic representations of individuals and populations as the
base-line representations and the mutating model switches between them when
appropriate. The population model requires the assumption that the likelihood
of an individual being at any given location within its ambit is represented
by its distribution function and that the population's contact with the
contaminant can be distributed through its members according to this function.

\section{Experimentation}

The comparison of the three modelling schemes is based on the following:
\begin{enumerateroman}
  \item the run-times for each representation
  
  \item the correspondence between the total contaminant loads across the
  biomass through time
  
  \item a comparison of variability in contaminant load through time amongst
  the representations
  
  \item and the sensitivity to variability in the nature of the plume.
\end{enumerateroman}


Our strategy is to first establish the equivalence of the results from the
mutating submodel representation and the individual-based representation. This
component of the study also provides us with data for comparing the relative
efficiency of the representations in terms of run-time. We generate forty
trials each simulating forty individuals for both the homogeneous
individual-based model and the mutating model. The homogeneous population
model is deterministic and so one run is sufficient.

A second set of trials compare the performance of the mutating model against
the homogeneous population model when the contaminant plume is elliptical.
Ideally, the results with an elliptical plume would be consistent across the
model representations, though the results will show this is not the case.

\subsection{The models of organisms}



The test models are composed of one or more submodels which run within a
simple time-sharing system. Each submodel runs for a nominated period of time
and passes control to the next submodel, very much like tasks running in many
modern computer operating systems.

The population-based and individual-based submodels have been kept as similar
as practicable in order to minimise the sources of divergence. Both of these
basic representations are used to model a group of organisms which proceed
around a common circular migratory path on an annual basis. While both
submodels use the same uptake-depuration model to calculate their contaminant
load, the calculation of the contact with the contaminant differs since the
interactions of an individual moving through a plume is, in some sense, an
integration across a path, while the population's expected interactions are
more analogous to the integration over the domain of the population.

The individual-based representation models the group as a set of individuals
(agents, or instances of submodels) which move in a directed random walk. At
each time step the trajectory of each indi{\tmem{{\tmem{}}}}vidual
incorporates a directional component toward the ``target location'' on the
migratory path which corresponds to the notional ideal centre of the group at
that time. Each agent maintains its own contaminant load, position and
velocity, and these attributes are independent of the state of the other
agents.

Our population-based approach models the organisms with a radially symmetric
distribution whose centroid lies on the migratory path. The density of the
simulated individuals around their target location arising from their movement
model closely follows the probability density function of a 2d normal
distribution. From this representation of the individuals' density, we define
the population's density, $\rho ( \tmmathbf{p})$, to be the value of the
standard 2d normal distribution with a mean of zero in each ordinate and a
variance. So our population's density at $\tmmathbf{p}$ is given by
\[ \rho (| \tmmathbf{p} |) = S_{_L }^{} \frac{1}{2 \pi \sigma^2} \exp \left( -
   \frac{p_x^2 + p_y^2}{2 \sigma^2} \right) \]


where $\tmmathbf{p} = (p_x, p_y)$ is the position in the population disk
relative to its centre, $S_L$ is a scaling parameter (equal to 1.015) which
ensures that the population density, $\rho$, integrated over the finite disk
with a radius, $\tau$, is one, and $\sigma^2$ is the variance for the
distribution. The effective radius of the population, $\tau$, was taken to be
three times the variance in the distance of the individuals from $\tmmathbf{m}
(t)$, which is about 9.5\% of the migratory radius. If the distribution of
individuals around the point $\tmmathbf{m} (t)$ and the distribution of the
population around $\tmmathbf{m} (t)$were not consistent, then the contaminant
exposure of the two submodels would be incompatible and the sort of mutation
error discussed in the introduction becomes significant.

In contrast to individuals, populations do not need to maintain position or
velocity since both are determined by $\tmmathbf{m} (t)$. They do, however,
maintain the number of organisms they represent and, in order to make mutation
simpler, populations maintain a list of contaminants. In the homogeneous
population model the length of this list never exceeds one, since the
population has only one mean, but in a mutating submodel, the contaminant
loads of the individuals are carried along and decayed in the usual way.

\subsection{The migratory circle and the plume}

The migratory path is the circular periodic function $\tmmathbf{m}$(t) which
has a radius of $R$ and a period $Y$. This path is used for all the variant
submodels of the system.

The plume is modelled as a circular or elliptical cloud of some contaminant
which intersects this path and has an intensity which varies sinusoidally at a
frequency which is relatively prime to the annual migration cycle.

Its centroid is positioned on the migratory circle. The plume is modelled as
an intensity at a location with an attenuation function. The intensity at a
point, $\tmmathbf{r}$, within the plume at time $t$ is given by
\[ I (t, \tmmathbf{r}) = \frac{1}{2} (1 + \cos (2 \pi t / p)) \exp (- \lambda
   \phi ( \tmmathbf{r}, \tmmathbf{m} (t))), \]


where $p$ is the period of the plume, $\lambda$ is a decay exponent, and for a
circular plume the distance function $\phi$ is taken to be $\phi (
\tmmathbf{a}, \tmmathbf{b}) = | \tmmathbf{a} - \tmmathbf{b} |$. The intensity
is made elliptical by instead taking $\phi ( \tmmathbf{a}, \tmmathbf{b}) =
\sqrt{( \tmmathbf{a} - \tmmathbf{b}) \cdot ( \sqrt[]{2}, \sqrt{1 / 2})}^{}$
since this transformation preserves the area of the plume. The effective
radius of the circular plume in the model is about 3.7\% of $R$.

\subsection{Contaminant load and contact}

The model for uptake and depuration of our contaminant is the same for both
representations. Initially a ``total contact'' is calculated for the time
step. As might be expected, the total contact for a population is calculated
in a different way to the total contact of an individual. In either case, this
resulting contact is fed through a standard uptake-depuration model
\[ d C / d t = u M - \lambda C \]
which is solved numerically with a fourth order Runge-Kutta algorithm for the
value of $C$ given a contact mass, $M$, and an initial contaminant value or
vector of values for $C$.

For individuals, the mass of contaminant which is available for uptake, or
contact, is taken to be the result of integrating the intensity of the plume
over the path of the individual, $\tmmathbf{P}_t$ to $\tmmathbf{P}_{t +
\delta}$,
\[ M = \int_{\tmmathbf{P}_t}^{\tmmathbf{P}_{t + \delta_{}}} I ( \tmmathbf{p})
   ||\tmmathbf{p}|| d \tmmathbf{p} \]
where we assume that the motion between $(t, \ell_t)$ and $(t + \delta_c, l_{t
+ \delta})$ is linear. Populations are somewhat more complex. Here, we
calculate the definite integral
\[ M = \int_{\tmmathbf{P}_t}^{\tmmathbf{P}_{t + \delta}} 2
   \int_{\tmmathbf{\Omega}} I ( \tmmathbf{p +} \tmmathbf{\omega}) \rho (
   \tmmathbf{p} + \tmmathbf{\omega}) d \tmmathbf{\omega} d \tmmathbf{p}^{_{}}
\]
where $\tmmathbf{\Omega}$ is an area over which we assess the effective area
of the population and $\tmmathbf{p}+ \omega$ denotes the area
$\tmmathbf{\Omega}$ translated so that its centroid corresponds to
$\tmmathbf{p}$. The contact equations are solved using a simple adaptive
quadrature routine.

\section{The mutating submodel}

The mutating model is formed by interleaving the population representation and
the individual-based representation. The transitions between a submodel and
its alternate representation are dictated by its proximity to the contaminant
plume. Initially the model begins with the group of organisms represented as a
population. The starting location is far enough away from the plume that there
is no chance of any interaction between organisms represented by the
population and the contaminant field. As the population moves about the circle
it reaches a point at which there is a likelihood that sometime in the next
few time steps an individual within its disk could conceivably encounter
contaminants, and at this point the population is disaggregated into submodels
representing individuals. Similarly, an individual-based submodel switches to
a population submodel when there is no possibility that it will encounter the
plume. These decisions obviously rely on knowledge of the system; specifically
the static location of the plume, the rates of movement and the length of time
steps.

The scenario presented exhibits traits that are common to a number of
situations which may pertain to real models. Many of the sources of
contaminants, such as stormwater outfalls and agricultural crops, are
relatively stationary and in some regards their ``plumes'' are reasonably well
bounded and predictable. The migratory path is a simple equivalent to some a
long term, periodic behaviour that brings the animals into contact with the
area of the plume, such as the seasonal movement from one foraging ground to
another, migration for breeding purposes, and even the long intervals of
sequestration exhibited by cicadas.

The contaminant levels of each individual-based submodels subsumed by the
population representation is maintained and updated appropriately. \ No
relative location information is maintained.

\section{Results}

The analysis in this section is based on two sets of simulations. The purpose
of each of these sets is to provide data for a comparison of run-times, the
equivalence (or lack of equivalence) amongst the models, and to examine the
robustness of the representations to changes in the configuration of the
plume.

The first set is comprised of forty trials of each of the homogeneous
representations (individual-based and population-based) and eighty trials of
the mutating model. Each of these trials tracked either forty individuals or a
population which represented forty individuals or a mixed system. In this set
of runs the contaminant plume was circular and its centroid was on the
migratory circle. This set of runs was conducted to establish the relative
cost in cpu-time of the three approaches and to experimentally verify (at
least to some degree) the equivalence which ought to exist between the purely
individual-based submodel and the mutating submodel.

The second set of runs consisted of eighty trials of the mutating model, and
one of the homogeneous population model. The principal difference between the
first and second sets is that the plume in the second set has an elliptical
footprint running along a tangent to the migratory path rather than the
circular plume of the first trial. Our initial analysis will only consider the
trials in the first set of runs; a comparison of the reponses to the
elliptical plume will be treated separately.

Both population-based and individual-based representations use a heuristic
which suppresses contact evaluation when they are sufficiently far from the
contaminant source. No other significant optimisations of the code have been
made, and as far as possible the same routines are used in both
representations. The motivation for this approach is to provide a common
baseline which we can use to measure the utility of the adaptive
representation.

To ensure that run-time comparisons are meaningful all of the simulations in
the first set of trials were run on the same computer. Each of the
configurations simulated forty organisms for twelve years.

\subsection{Contaminant load correspondence between representations}

The set of trajectories arising from the various representations aren't
directly comparable. Our individual-based representation produces a time
series of contaminant levels for each individual, while the population
submodel produces a ``mean load'' across the whole group of entities it is
representing. The mutating submodel sits between the two, sometimes producing
individual time series and sometimes mean time series for varying parts of the
population. We denote representations by a subscript $r \in \{i, m, p\}$, so
that $C_{r k j} (t)$ is the time series associated with individual $j$ in
trial $k$ of representation $r$, $C_{r k} (t)$ is the mean over all the
individuals in the indicated representation and trial, and $C_r (t)$ denotes
the mean of $C_{r k} (t)$ across the $k$ trials for the indicated
representation. In order to compare the dynamics of the system we generate
mean time series for each of the $k$ trials in the individual-based and
mutating sets, $^{} C_{i k} (t)$ and $^{} C \tmmathbf{}_{m k} (t)$, paying
special attention to generating the correct mean in the mutating submodel from
time steps which have a mixture of individual trajectories and mean
trajectories from population-based representations. Each of mean time series,
$C_{r k} (t)$, correspond to the mean contaminant load of the population, $C_p
(t)$, produced by the population submodel; averaging them -- constructing
\[  C_r (t) = \frac{1}{k} \sum^k_{j = 1} C_{r j} (t), \]


where $r$ is one of `$i$' or `$m$' -- is equivalent to running many stochastic
trials and averaging to fit the population submodel.

Using $^{} C_{i k} (t)$, $^{} C_{m k} (t)$ and $^{} C_p (t)$ we find the
maximum value attained for each representation, $\hat{C}_r$. We are also
interested in the mean value across time of each
representation,{\hspace{<htab|0>}}
\[ \bar{C}_r = \frac{1}{T} \sum_{t \in T} \]
These quantities are listed in Table \ref{MaximaMeans}.

\begin{table}[h]
  \begin{tabular}{ccc}
    Time series & $\hat{C}_r$ & $\bar{C}_r$\\
    $C_i$ & $0.1787$ & $0.0390$\\
    $C_m$ & $0.1821$ & $0.0392$\\
    $C_p$ & $0.1387$ & $0.0350$
  \end{tabular}

  \caption{Maxima and Means\label{MaximaMeans}}
\end{table}

\subsection{Contaminant load variability}\label{LoadVar}

We calculated measures of variability in the time series using the aggregated
time series $^{} C_{i k} (t)$ and $^{} C_{m k} (t)$ and their respective means
across the $k$ trials, \ $C_i (t)$ and $C_m (t)$. We'll take $T$ to be the
total number of time steps taken, and we take
\[ \hat{\sigma}_{a b} = \max_{t \in [1, T]} \left[ \frac{1}{k} \sum_{j = 1}^k
   (C_{a k} (t) - C_b (t))^2 \right]^{1 / 2} \]
and
\[ \sigma_{a b} = \left[ \frac{1}{T} \sum_{t = 1}^T \left[ \frac{1}{k} \sum_{j
   = 1}^k (C_{a k} (t) - C_b (t))^2 \right] \right]^{1 / 2}, \]
to be the maximum root mean square error and the average root mean square
error. Clearly we can write $\sigma_{r r}$ as $\sigma_r$ without introducing
ambiguity, and similarly for $\hat{\sigma}_r$. The values for these measure of
of variability are presented in Table \ref{LoadVarTbl}.

\begin{table}[h]
  \begin{tabular}{cccc}
    StdDev & $r = i$ & $r = m$ & $r = p$\\
    $\widehat{\sigma_{}}_{i r}$ & $0.0083$ & $0.0084$ & 0.0534\\
    $\sigma_{i r}$ & 0.0024 & $0.0024$ & $0.0096$\\
    $\widehat{\sigma_{}}_{m r}$ & $0.0090$ & $0.0090$ & 0.0538\\
    $\sigma_{m r}$ & 0.0024 & $0.0024$ & $0.0096$
  \end{tabular}{\hspace{<htab|0>}}{\hspace{\hfill}}
  \caption{Deviations amongst the model runs with respect to a given
  mean\label{LoadVarTbl}}
\end{table}

\subsection{Sensitivity to the shape of the plume}

We will use the same notation as Section \ref{LoadVar} for the data derived
from the circular plumes, while we will add a prime symbol to the data derived
from the elliptical plumes. Thus, the mean value time series for the mutating
submodel with elliptical plumes would be denoted $C'_m$ and the mean value of
that time series is $\bar{C}'$.

The data for the circular plume is presented in Table \ref{Symplume} and the
data for the elliptical plume is presented in Table \ref{Asymplume}.

\begin{table}[h]
  \begin{tabular}{ccccccc}
    Time Series & $\hat{C}_r$ & $\bar{C}_r$ &  & StdDev & $r = m$ & $r = p$\\
    $C_m$ & $0.1738$ & $0.0392$ &  & $\widehat{\sigma_{}}_{m r}$ & $0.0088$ &
    0.0535\\
    $C_p$ & $0.1387$ & $0.0350$ &  & $\sigma_{m r}$ & $0.0024$ & $0.0098$
  \end{tabular}
  \caption{Circular plume results\label{Symplume}}
\end{table}

\begin{table}[h]
  \begin{tabular}{ccccccc}
    Time series & $\widehat{C'}_r$ & $\overline{C'}_r$ &  & StdDev & $r = m$ &
    $r = p$\\
    $C'_m$ & $0.1856$ & $0.0394$ &  & $\widehat{\sigma_{}}_{m r}'$ & $0.0087$
    & 0.0616\\
    $C'_p$ & $0.1763$ & $0.0445$ &  & $\sigma_{m r}'$ & $0.0025$ & $0.0092$
  \end{tabular}
  \caption{Eliptical plume results\label{Asymplume}}
\end{table}

\subsection{Run-time}

Each run collected data regarding the amount of time spend in different parts
of the submodel. As a basis of comparison, the overall amount of time spent
running on the cpu (linux/unix ``cpu seconds'') for the whole run is the most
important figure, though the time spent in other parts provides illumination
into just where the effort is concentrated. Predictably, most of the effort is
in calculating contact and updating contaminant loads.

The optimisation of supressing the contact calculations when a population is
outside the area of potential contact seemed to make very little difference to
the run-time of population submodel (of the order of 3\%), and seems unlikely
to make a great deal of difference to the mutating submodel. In the case of
the purely individual-based submodel, this sort of optimisation is likely to
play a much bigger role in that a three second penalty would be multiplied by
the number of entities simulated.

The population submodel ran for 98.7 cpu seconds. This submodel is
deterministic and the amount of cpu time used is very stable, so only a single
run is considered for comparison. The purely individual-based submodels took
just over a mean time of 4205 cpu seconds with a standard deviation of barely
more than 16 seconds and the mean of the mutating submodel's run time was 1157
cpu seconds with a standard deviation of slightly over 11 cpu seconds.



\section{Discussion}

The data establish the relative speeds of the three approaches and support the
assertion that the mutating submodel possesses the same essential dynamics as
the purely individual-based representation. The essential equivalence of the
individual-based and mutating representations is well supported and there is a
significant benefit in run-time with the mutating submodel when compared to
the individual-based submodel. The population submodel, while fastest by
several orders of magnitude, showed responses to the two distinct contaminant
plumes which were significantly different to those of the other two submodels.

The divergence between the population's density function and the observed
distribution of individuals in the individual-based representation is
potentially a source of model error. Though the actual distribution of
individuals is slightly skewed by their motion along the migratory circle, it
was very close to a two dimensional normal distribution and a standard 2d
normal distribution was chosen to represent the density function of population
with its variance taken from the observed individuals. Such a distribution
corresponds to an ideal situation -- in practice there may be a number of
environmental influences which make a such a simple population distribution
unrealistic: in the case of a coastal population, exceptions might include
mud-flats, sand bars or islands.



\subsection{Mutating models}

The mutating model tests whether it may be more efficient in terms of time or
representational accuracy to change the representation of components of the
model in response to the local states of its submodels. The specific example
we are dealing with is very simple, but it possesses many of the properties
which the general problem of an {\tmem{ad hoc}} animal/plume interaction might
be expected to exhibit, namely the tests for coincidence, generating a random
walks, and evaluating the uptake and decay of contaminant loads. Over a long
period of infrequent contact, we would expect quite a lot of time to be spent
generating random walks and evaluating the contact and uptake in places where
there is no likelihood of contact between animals and the plume. This strategy
of changing representation is fundamentally about managing the often competing
demands of representational accuracy, mathematical tractibility, and run-time.

The mutating submodel is an ensemble of cooperating models working together
like a team in a relay race. Each of the representations has its own strengths
and passes the baton to another at an appropriate time. So each of the
submodels must have mappings between other representations which preserve as
much information as possible without imposing excessive burden. As a first
step in construction we must determine what information should be maintained
in each of the representations to ensure that the migration from one
formulation to another doesn't introduce unreasonable error and how that
information should be maintained. In our example, the {\tmem{significant}}
information which the individual-based representation possesses which the
population representation does not is the variation in contaminants amongst
the individuals within the group. We assume that the role of the individual
locations of these individuals in this model is not important over a large
portion of the state-space since the interval between conversions from
individuals to aggregates is large enough to ensure a thorough randomisation
of their relative position.

The state spaces of our base-line representations differ, in contrast to the
example of Newtonian and Relativistic motion. The natural state space of the
individual based model must incorporate position information lacking in the
population's state space, since the population's movement is determined wholly
by $\tmmathbf{m}(t)$ and the individuals each have their own path. This is
quite different to adaptive models where scales or step sizes are adjusted to
improve the efficiency of the algorithm. \ Not only does the algorithm change
in this model, but the domain and range of the system changes as well, and the
mapping between these state spaces should be considered carefully.

The initial premise was that an individual's trajectory through its
environment played a significant role in its contact with contaminants, and
that the population model didn't capture this contamination load as well. The
implication is that the contaminant loads of the individuals must be preserved
through model transitions, so in this regard the state space should remain the
same, though our implementation of the uptake and decay of the contaminant
load of populations has to operate on a list of contaminants.

The transition from individuals to population and population to individuals
involves the loss and reconstruction of this fine-scale position data. The
assumption that we can reconstruct a plausible position for each individual
from the population's distribution function is tenable because there is no
behavioural change associated with contaminant load -- a contaminant that made
an organism sluggish would obviously skew the distributions of both the
population as a whole and the distribution of intoxicated organisms within the
population.

In the presented model, individuals are mapped into a population by adding
their contaminant level to the contaminant list of an appropriate population
after their movement and contaminant decay have been calculated. When a
population submodel is mapped back to an individual-based representation, new
individual-based submodels are created with appropriate contaminant levels \
and random locations which are chosen to be consistent with the distribution
for the population around its centroid. This mapping preserves the contaminant
load present in the system and the likelihood of an encounter between a
contaminated individual and new contaminants. In this way the distinct
contamination levels within the system are maintained with a minimum of cost.

Some overhead is incurred in managing the swap from one representation to
another. The actual conversion is done by a routine which is integrated with
the scheduler which manages the transition from one time-step to the next.
Each of the submodels is able to indicate to the scheduler that a change in
representation may be reasonable when it determines that it has left its
domain of efficiency.

This illustrates the two basic ways of managing the transitions between state
spaces: by adjusting the algorithmic representation of the a submodel to
maintain essential data, or by dynamically generating missing data at the
point of transition. \ In our example model both of these mechanisms are
relatively straightforward, but it is easy to see how this may not always be
the case.

\subsection{Transition heuristics}

When {\tmem{do}} we swap from one representation to another? More
fundamentally, {\tmem{how}} do we decide? The contraints we impose on this
example model are quite stringent so the heuristics we can use to decide when
to change representation may assume a great deal more than might be the case
in a general situation: we know where the contaminants are at all times, the
movement of the populations and the individuals are controlled, and the
behaviour of the simulated organisms does not change through the run. In the
general case, the heuristics might be much more complex and require a great
deal more data. Deciding when to change the representation is central to the
optimisation of the system: changing at the wrong time might sacrifice
accuracy, efficiency or both.

In our case, the principal issue is whether or not an agent (population-based
or individual-based) may encounter contaminants: in our case it is simple,
there is a single source and the mean movement toward or away from that source
is quite predictable. As a population enters the region of contaminants we can
immediately determine that it will consistently encounter contaminants for
some non-trivial period, and a mapping is called for.

The test which determines if an individual is mapped into a population is
similarly straightforward: an individual is mapped into a population if it is
far enough outside the region of contaminants that there is no chance that it
might encounter them in the next few time-steps, and it is close enough to a
population. If there is no suitable population, a new one is created in order
to accomodate it. This spatial constraint is stringent enough, when combined
with the movement constraints, that the system hasn't needed to create more
than one population in any of the annual cycles in the trials.

The transition rules in this submodel, from population to individual or from
individual to population, can be based solely on the distance to the
contaminant source, the time-step and the speed at which the simulated group
of organisms moves around its annual path. Our heuristic could be generalised
to systems where there were multiple, dynamically instanciated sources, when
the sources of contamination could be determined at the beginning of a
time-step. This particular submodel attempts to present a greatly simplified
version of the general case -- if the hypothesis were to fail in this case it
would be reasonable to assume that it would fail in most cases.

\subsection{Means, maxima and variation}

In terms of the formulation of the mutating submodel, it is difficult to see
how there could be a significant difference between its contaminant loads and
those of the the purely individual-based submodel. The means, maxima and the
various calculations of deviation between the individual-based and mutating
submodels were very close and this supports the belief that they can be
treated as essentially equivalent models of the contaminant load in the group
of simulated organisms. Both of these representations, however, differed
noticably from the population submodel; the population's mean value,
$\bar{C}_p$, was about 10\% lower than that of the either $\bar{C}_i$ or
$\bar{C}_m$, and the statistics on the variation between representations
($\sigma_{a b}$, and $\hat{\sigma}_{a b}$) indicate that there is very little
difference between the data produced by either the base-line individual-based
model and the data produce by a mutating model. \ The population model, in
contrast, produces means and maximum values are significantly lower (11\% and
22\%) than those of the \ individual-based model.

The discrepancy between the population-based uptake and individual-based
uptake is enough to suggest the following possibilities:
\begin{itemizedot}
  \item the formulation of the population distribution was not consistent with
  the spatial dynamics of the individuals,
  
  \item the tuning of the population submodel with respect to the individuals
  was inadequate,
  
  \item the model of interactions between the plume and the population or the
  plume and the individuals was not appropriate
\end{itemizedot}
This is the sort of situation discussed in section \ref{Plume1}: the
fundamental submodels exhibit different dynamics over the domain of the
simulation, and this is one of the niches which a mutating approach might be
most advantageous. It is difficult to accurately describe a wild population's
distribution across the whole of its domain -- changes in the behaviour of
individuals (arising from predation, drought or other stressors) may engender
quite different population distributions, so even in the case of a purely
population-based representation an argument can be made for changing the
underlying representation based on the local state.

\subsection{Representational sensitivity to plume distribution\label{Plume1}}

The issue of whether the configuration of the plume has an effect on the
results of a simulation depending on the representation of the organisms is
important. It is easy to imagine calibrating a population-based representation
with data from fine-scale individual simulations in order to make the most of
the economies a population-based representation affords, but this may not be a
good idea if there is poor correspondence between the two submodels over their
domain.

A set of trials with an elliptical plume were run to generate a set of data
for comparison against the the data from the trials with a circular plume. Our
analysis in these trials is based on the data from the first series of trials
and from eighty trials of forty individuals with a mutating representation and
a corresponding population-based trial with each trial covering twelve years.
The elliptical plume covers the same area as the circular plume and the
integral over the area of the plumes are the same, up to the numerical error
in the quadrature.

The data from the mutating model in the set of trials with elliptical plumes
had means, maxima and deviations which were consistent with the set of trials
which used circular contaminant fields. This matches our expectations since
the total areas covered and the total contaminant load over the area are the
same for the two representations, and the speed of the agents through the
plume is relatively slow. In contrast, the circular and elliptical trials of
the population submodel had markedly different maxima and mean values through
time.

The data suggest that the individual-based representation is reasonably robust
with respect to the plume shape (at least in the context of this experiment).
The population submodel did not behave as expected and it may be that the
model of the distribution of individuals which formed the basis of the
population submodel may have systematic problems with the population wide
sampling of the plume -- the increase in mean and maximum levels when the
plume was elliptical and oriented tangentially along the track of the
population suggests that the centre of the population is oversampled. The
implications of this are that we must choose our population distribution
carefully, and a simple normal distribution about a centroid may not behave
well with respect to the plume dynamics.

Set against this, the numeric modelling of populations is well established. It
is quite likely that the individual-based submodel would fail to meet
expectations with regard to the dynamics of recruitment and mortality, had
those aspects of the life history of a population been included in the model.
In such a situation, the mutating submodel would have performed best overall
-- largely maintaining the fidelity of each of the representations across the
state-space.



\subsection{Run-time}

The population submodel is clearly much faster than either of the other two
representations, and for situations where we can reasonably make mean-field
assumptions and our distribution model fits well with the observed data it is
likely that the economies afforded by the speed of execution and the well
established mathematical understanding of such models will vastly outweigh
benefits of individual-based representations. In situations where mean-field
assumptions occasionally fail to hold it is similarly clear that there are
benefits to using mutating submodels. Purely individual-based submodels entail
significant overhead, which in our context contributes nothing to the outcome.
In a richer environment with diverse responses to the local state, an
individual-based submodel's `non-productive' overhead may be competitive with
the overheads associated with the management and transition of mutating
submodels.

It is difficult to decide what an optimal representation of a system is --
not only is the trade-off between run-time and resolution a dilemma, but we
then have to consider (at least in the case of stochastic models) whether to
go for minimal error, or to maximise the number of trials in the time at hand.
In the case of error versus number of trials, we can make an informed decision
on where that balance lies if have a good estimate of how closely the model
tracks ``truth'' but this is often not possible, particularly when the subject
of the study is poorly observed.

\section{Conclusion}

The case for considering modelling schemes where the representation of a
system changes in response to its local state seems to have merit. There are
clear advantages in run-time relative to homogeneous individual-based
simulations. \ In contrast there would seem to be little to gain in run-time
with a shift from population-based representations to individual-based
representations, but in situations where there are behavioural changes --
particularly changes which affect the distribution of individuals -- the scope
for increasing the accuracy of the models is attractive.

The principle source of extra overhead in this modelling structure is in the
scheduler which manages the multiple submodels. \ In the homogeneous
population model, there is only one submodel running, and the extra cost of
the more general scheduler over a simple iterative loop is negligible.
Similarly the overhead in the homogeneous individual-based model is comparable
to an additional iterative loop over the agents. Neither contribute a
significant amount to the total time spent running a trial; both the solution
of the uptake-depuration equation and the calculation of the contact dominated
the run-time in both homogeneous models.

This simple model demonstrated that changing the representation of a system
from one form to another can provide a mechanism for increasing the efficiency
or accuracy with only a little extra effort. A more challenging avenue for
study would be a similar model which included population recruitment and
mortality -- particularly if there were sub-lethal effects associated with the
contaminant load which changed the behaviour or fecundity of the simulated
organisms.

Even if we can count on the regular doubling of computational capacity which
we've enjoyed for so long, the magnitude of the problems we consider seems
likely to grow as fast as our capacity and possibly faster. \ In our
experience of large scale marine ecosystem modelling, the size of the system
considered is growing much faster than the computational capacity. Even for
small systems the possibility of adjusting the representation of submodels to
optimise the accuracy of the model as a whole has great appeal. Mutating
models may provide an effective means of concentrating the use of
computational capacity where it is most needed. \





\end{document}
