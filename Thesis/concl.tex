
\chapter[CONCLUSION]{Conclusion}\label{concl}
%If you laid all the modellers in the world end to end, you'd have had
%too much time on your hands.

\subsection{Modelling over the past few decades}

The domain models are expected to cover is increasing in breadth and
complexity --- ecosystems models, for example, may now be called on to
represent large areas, and significant numbers of species.  This is
not inherently intractable; equation-based modelling is well suited to
the modelling of such systems, but we are also often required to
include subsystems where equation-based models are a poor fit. Hybrid
models, which incorporate both equation-based representations and
agent- or individual-based models for those components they are best
suited, offer a way to deal with many of these situations.  This
solution isn't adequate where the nature of the interactions changes
the dynamics of the system---classic examples being the effects of
parasites or contaminants which alter behaviour or reproductive
success. Chapters~\ref{modelefficiency} and~\ref{adaptiveselection}
showed that changing the representation of a system using a selection
strategy based on its state may have benefits both in terms of
representational fidelity and in computational cost.  

\subsection{Possible, but not implemented}

Record of transitions---indicative of the sensitivities of the
system, 




\subsection{Deterministic strategies}
% Like Chapter~\ref{modelefficiency}
benefits---fast

copes with imperatives

rigidity

\subsection{Adaptive strategies}
% Like Chapter~\ref{adaptiveselection}
not without overhead

can handle imperatives, but still overheads (associated with
assessment and mapping to and from assessment space)

flexible


\subsection{Combinations of the two}

\subsection{Parallelism and Distributed models}




\subsection{Into the future}

Use of region partitioning in the state space

nearest neighbours

interpolation






The likelihood is that models will continue to increase in the scope
of the domains they model as well as the level of detail they are
required to simulate.  There are issues that arise from both of these
eventualities: as the scope of a  model's domain increases, the computational
requirements of the model typically increase much more rapidly---adding new
species or doubling the area may increase the number of interactions
by orders of magnitude. While quantum processors may help with some
aspects of simulation (such as path resolution or prey selection) to 
ameliorate this, it seems unlikely that they will provide a magic
bullet; in modelling, past experience suggests that the problem any new model
addresses will grow to point that it is just barely tractable. If this is
true, no real technological advance or methodological insight (such as
changing representations!) is really going to do much more that shift
the boundary: modellers will still sit impatiently waiting for the
model run to finish, wishing they had more memory or a faster CPU.

Perhaps more significant is the effect of increasing demand for
process resolution (``realism'').  Clearly faster hardware or more
efficient algorithms will help address this in a more sustainable way,
but it isn't always the case that a more resolved model will improve
the correspondence between the model results with the trajectory of a
real system.  Equation based systems, for example, are usually
constructed with parameterisations that incorporate the naturally
occuring aggregation of data that time and space provide.
Constructing an individual-based model from parameters obtained by
observing the activity of individuals may not produce, the sort of
large scale dynamics observed in the analytic model or in vivo, and an
individual-based model using population-scale parameterisations may
fail to exhibit the dynamics observed in individuals.

Models with embedded submodels which have naturally different scales
(temporally or spatially) need to be able to ensure that the
interactions between elements of the models are well founded: it may
be perfectly sound to simulate juvenile cicadas in a forest stand
using timesteps of seven or thirteen years for most of their life
cycle, but such a timestep will fail to capture the dynamics of the
population once they emerge.  There is clearly a need for approaches
which can adapt to dramatic changes in behaviour, time scales, and
spatial range.

%% A number of predators (sharks, birds, cetaceans) may
%% travel quite significant distances to take advantage of infrequent
%% (though often annual) events, such as the hatching of turtles or the
%% departure of penguins from rookeries). These cases may not require a
%% complete change in representation, but there is a significant change
%% in their foraging strategy. These examples are readily dealt with by
%% incorporating alternate code paths when special conditions pertain,
%% but a more general approach allows for models which can 



The (HERE WHAT?) in this work addresses the problems which arise when dealing
with complex systems with components with dynamics which are sensitive
to the state of the system.  Models can be constructed so that their
components can adapt to the most appropriate form for the state of the
model through the whole simulation. In practice, it is possible to
choose a strategy which optimises for any appropriate measure ---
speed, fidelity, or the cost of real-world management, for example.

