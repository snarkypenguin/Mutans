\begin{document}

%% These are defined at the end of the mathphdthesis.sty
\titlepg
\signaturepage
\altcopyrightpage
\abswithesis
\ackpage
% Set page numbering to arabic the first time we commence a chapter.
% This is required to get the page numbering correct.
\pagenumbering{arabic}

% Note that the text in the [] brackets is the one that will
% appear in the table of contents, whilst the text in the {}
% brackets will appear in the main thesis.
\chapter[INTRODUCTION]{Introduction}\label{intro}

\section{Overview and introduction}

The primary aim of this thesis is to demonstrate an approach to
constructing computational models which can be used to effect changes
in the mathematical or algorithmic structure of model of complex
systems in response to changes in their status. A simple
uptake-depuration model of organisms which migrate through a
contaminant field is presented in Chapter \ref{modelefficiency} to
explore the fundamental principles and to demonstrate the utility of
such a mechanism. The third chapter develops a mathematical structure
with properties that make it a useful mechanism for representing the
complex states of these structurally adaptible models. A number of
useful functions are defined, and the basic properties of the
structure are proved. Chapter \ref{adaptiveselection} uses this mathematical
structure for encoding the complex states of a model and its
components in a ``thought experiment'' which illustrates the dynamics
which might arise in a simple model. 

Chapter \ref{explicitmodel} provides a detailed discussion of a real
implementation of the thought model used in Chapter
\ref{adaptiveselection}. The chapter briefly describes the modelling
framework used to implement the model, and then addresses the details
of the implementation. \textbf{Using ODD?} {\emph The body of code is included in
the Appendix.  If we include it all it is 500pp}


In this thesis, we will use \emph{model} to refer to a (possibly
trivial) collection of of components which, together, simulate a
system. The term \emph{submodel} will generally refer to a largely
discrete component of a model (arguably a model in its own right).
More specifically, we will consider submodels to be the abstract or
\emph{in potentia} representations, and \emph{agents} to be the
\emph{in esse} instances of submodels.

Traditional models of complex systems incorporate alternative code
paths or expressions to deal with fundamentally different dynamics
which may be required within a component of the model by testing for
conditons which arise during a simulation and executing an alternate
code path within the component. Examples of this approach can be found
in models implemented in a number of popular modelling frameworks
{\citep{netlogo,swarm,repast,mason} and in my previous work
  \citep{nws,ningaloo}.

In contrast, the models used as examples in this thesis deal with such
changes by changing the component used --- a number of agents
(instances of submodels)\footnote{} associated with individual contact
with a contaminant plume might be substituted for a population agent
which has no code path of its own to deal with contaminant uptake or
depuration.

Representing the state of the model, either as a whole or of its
constituent parts, is not simple: not only may the components of the
model vary through time, but their own dependences on other components
may change as the state of the model changes: a model which contains a
``whale spotting'' tourism venture may follow the activities of whales
at particular times of the year, but be utterly indifferent to them at
times when there is little likelihood of their presence in an
accessible location. Similarly, the association of entities
represented by a population-based model needs to be maintained if the
population disaggregates into super-individual-based or
individual-based agents.

These factors make a simple vector-based encoding mechanism for the
states of a model and its components awkward, thus we turn to a metric
space over a \rng\footnote{To avoid confusion, we take \rng\ to denote
  a ring without a multiplicative indentity element.} of trees with a
finite number of weighted, labelled nodes. This simplifies the
comparison of possible component mixes for a given global state, and
makes available any algorithms (such as for clustering) which depend
only on ring properties, excluding those that need a multiplicative
unit.

Any attempt to assess the relative merit of a number of possible
component mixes for a model relative to its current state must have
some way of incorporating the necessary dependencies arising from the
effects of the current state on the , and be able to calculate (in
some sense) the incremental costs of change.


\section{Historical work}

\texttt{Probably need more historical work here}
    
Many individual-based models incorporate data which represents
environmental characteristics and influences the behaviour of of the
individuals simulated. \citet{Botkin72:2} and \citet{deangelis1978model}
are important early example which model individuals in in the context
of spatially explicit environmental gradients. In both cases, the
subjects of the simulations are represented as individuals, and the
environments are relatively static (though interactions, such as
shading in JABOWA, also played a role).  Even as formulated, we can
consider the environments in these models as static models in their
own right: from its early inception, the individual-based modelling
approach implicitly couples different styles of model components.

The coastal marine ecosystem model in \cite{Gray06:1} used different
representations for the agents which represented the organisms
comprising the benthic habitat based on their life-stage, though the
nature of this change was configured during the compilation of the
model.  Here, juvenile biomasses could be represented by polygonal
clouds which changed shape as they were advected, and adult stages
could be represented by several different submodels which might be
optimised either for speed or for spatial fidelity.  This model was in
most other ways similar to conventional agent-based modelling of the
time.

%\citea{gross2002multimodeling} describes using this approach as a
%basic element underpinning their model, ATLSS, of the Everglades
%region in the U.S.A.

\cite{bobashev2007hybrid} describes a model of epidemic simulation in
which the representation of populations or portions of populations are
decided based on the number of infected individuals relative to a
nominated trigger value. This model demonstrates that there is an
advantage to changeing representation in terms of computational
efficiency and the fidelity of the model. The model in Chapter
\ref{modelefficiency} is similar: the rule governing the switching
from one representation to another occurs when some monitored quantity
(the number of infected individuals or the proximity to plume, for
example) crosses a nominated boundary.  The primary operational model
underpinning the work in \cite{Fulton2009crossingscales} takes a step
further, migrating whales change their representation according to
their putative geographic location, specifically whether they were in
the model domain or outside it. This model also had to consider the
linkages between agents: agents representing predators, wildlife
management and tourism operators had to be aware of the presence of
``in domain'' whales.

\section{Structure and function}











%% This new paragraph shows how to set \index{index items}index items
%% and \index{index items!subindex items} subindex items.

%% \subsection{New Subsection}
%% Here's a subsection with some simple maths $a^2+b^2=c^2$.

%% \subsubsection{subsubsection}
%% Here's a \index{subsubsection}subsubsection...oooooooohh....wow
%% wee!!!!!!

%% \newpage
%% Some more text to check indent and show how references work
%% \cite{Williamson:STS}.

%% Here's how we place a figure (Figure \ref{fig:utas}) on the page.
%% \begin{figure}[hbtp]
%% \begin{center}
%% \scalebox{.30}{\includegraphics{Utas_vert_BW}} \caption{
%% \label{fig:utas} The UTas logo}
%% \end{center}
%% \end{figure}

%% And finally, here's a table example (Table \ref{tab:taba}).
%% \begin{table}[hbtp]
%% \begin{center}
%% \begin{tabular}{|r|r|r|r|r|}
%% \hline
%% $n=$&2&3&4&5\\
%% \hline
%% $c$ (rad/day)&1.67&0.52&0.06&-0.17\\
%% \hline
%% period (days)&3.75&12.00&100.00&37.50\\
%% \hline
%% \end{tabular}
%% \end{center}
%% \caption{\label{tab:taba}A simple table}
%% \end{table}
