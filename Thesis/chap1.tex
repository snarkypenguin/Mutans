%% These are defined at the end of the mathphdthesis.sty
\titlepg
\signaturepage
\altcopyrightpage
\abswithesis
\ackpage
% Set page numbering to arabic the first time we commence a chapter.
% This is required to get the page numbering correct.

\pagenumbering{arabic}


% Note that the text in the [] brackets is the one that will
% appear in the table of contents, whilst the text in the {}
% brackets will appear in the main thesis.
%\setcounter{page}{2}

\chapter[INTRODUCTION]{Introduction}\label{intro}

\typeout{Chapter 1: Overview and introduction}
\section: Overview and introduction{Overview and introduction}

The term \emph{simulation modelling} covers an incredibly broad
domain. The models which simulate ocean circulation are quite
different from individual-based models of foraging ants or timber
growth in a forest, but they share common ground. Simulation models
are likely to be computer-based models which attempt to capture the
dynamics of a system in order to predict or analyse the state of the
system. More importantly, The systems we model are likely to be
important to us and we model them in order to develop our
understanding of the dynamics which govern them.  While there will be
a bias toward using examples and a narrative based on the modelling of
human-ecosystem interactions in this thesis, the central applies well
to almost any domain. The work which lead to the ideas behind this
thesis was conducted over many years modelling the impact of human
activity and testing, by simulation, different ways of managing that
impact.

In the last thirty years, simulation modelling has become a central
tool in studying, predicting and attempting to manage the environment
we live in. Simulation models are used to assess the likely impact of
development, of management strategies, and of climatic changes.  Both
the physical domains modelled and the complexity of the biotic and
abiotic systems represented are increasing in size and complexity, and
with this, our ability to comprehend these models in their entirety
diminishes. Indeed, the last major model I worked on had a word count
comparable to that of the entire ``Harry Potter'' series\footnote{A
model of the human-ecosystem interactions on the NW coast of Western
Australia (\cite{Gray2014}) with a ``\texttt{/usr/bin/wc}'' word count
of somewhat more than 1.3 million words, compared to just over one
million words in J.K. Rowling's ``Harry Potter'' series.}; this was
more than twice the size of its predecessor and more than four times
the complexity.  For most modellers, a corpus this size exceeds our
ability to maintain a fine-scale understanding of all the component
parts, much less to anticipate all of the possible interactions which
might be possible.  Software engineering principles attempt to
mitigate this in a number of ways, but these are primarily about the
\emph{correctness} of the code, rather than the \emph{appropriateness} of the code,
and very few projects have the scope, resources or skills to engage in
the formal proof of the correctness of the program as a whole.

Models of complex systems are vulnerable to a number of potential
problems. In the case of models in population biology or ecosystem
simulation, population levels may vary widely, and the inability of an
analytic model to deal appropriately with an individual's influence on
the population may become an issue at low population levels. At the
other end of the spectrum, individual-based models often have
difficulty with small errors in their fine-scale representation
dominating the the results at high population numbers.  This
corresponds to the disjunction between continuous processes and
discrete processes.

In these systems, it would be wonderful if the simpler systems which
comprise them were easily partitioned, but---at least in ecosystem
modelling---they usually seem interact directly or through indirect,
sometimes cryptic ways. The challenge, then, is to construct our
models with submodels of these simpler systems so that each submodel
is as close to ``independent'' as we can make it, and to be able to
tie the ensemble together in a reasonable facsimile of the system we
wish to simulate. This strategy works reasonably well in a large
majority of cases, and a modular approach also underpins our approach
to the design of complex machinery. A critical difference is that
engineers have a great deal more control over the components which
make up an actual machine than ecosystem modellers, for example, have
over the ocean or the biotic and abiotic environment in an ecosystem.

Modellers \emph{do} have some advantages over engineers,
however. In the domain of modelling we have the scope to employ
techniques which are not typically possible in physical systems: we
are able to change how we represent a subsystem within the model
on-the-fly, in order to maximise some objective.  Of course there
are common machines that \emph{do} change their physical
characteristics; examples might include aircraft which change their
physical shape according to needs (like the Concorde) and modern
hybrid-electric vehicles which only run their engines either when
their battery reserves are low, or they are under unusual
load. Typically, these machines can only change in relatively
constrained ways; a closer match to an adaptive model might be a bus
that is able to automatically adjust its length and seating
arrangements to accomodate the needs of its passengers and their
cargo.

Discrete changes in the behaviour of a system, or part of a system,
are commonplace. The scales of systems that exhibit switching
behaviour range from a molecular level, such as the behaviour of water
as it freezes, through to climatic changes.  The discussion of
``tipping points'' has increased dramatically in the last decade
\cite{bhatanacharoentipping}, indicating a broad recognition that
systems' dynamics can switch from one mode to another.  In this
context, consider the assumption that the cumulative effect of
individual variability tends toward a useful mean.  For \emph{many}
systems, the error arising from the mixing assumption is small;
if, however, there are processes modelled which apply to the subjects
unevenly or asymmetrically, then modelling based on this assumption is
likely to give an unsatisfactory result.  Sometimes the unevenness
arises from the nature of the processes themselves; contaminant plumes
or spills are usually localised and may have quite erratic footprints,
for example, and there may be physical constraints may effectively
preclude interactions with part of a population perhaps limited by
gape.  A more subtle source of heterogeneity occurs when the behaviour
of individuals within the population is not uniform, such as may occur
in a population containing parents with dependent young. These
examples are fairly straightforward -- while there is asymmetry, there
is little risk of significant feedback.

More complex interactions occur when the extent of the effect of a
process is associated with the spread or reproduction of
organisms. The effects of \emph{T. gondii} on
rats \cite{berdoy2000fatal} is well known, and the positive
reinforcement on the spread of \emph{T. gondii} afforded by the rats
loss of fear of cats leads to greater potential for the pathogen to
infect more rats.  \Citet{dobson1988population} investigates the
population dynamics of these kinds of interactions, and present an
approach to incorporating these effects into analytic
models. \cite{dobson1988population} explored the population dynamics
of parasite-host systems where the parasites influenced host behaviour
found that the reproductive capacity of the host populations could be
significantly modified by the behaviour altering
parasites.  \cite{poulin1994meta} assesses the effect on host
behaviour in a number of host-parasite pairings, and found that the
parasites had a significant effect on the behaviour of their hosts.
In the cases addressed in these papers, the process in question is
predation, and the infected individuals are often either
disproportionately preyed upon \cite{dobben1952food}, or involved in
the parasite's reproductive cycle.

Environmental conditions and contaminants may also change the
behaviour of animals in affected areas. \cite{de2011reduced} finds
that reduced seawater pH disrupts a hermit crab's ability to assess
the the relative merit of shells, thus reducing its own reproductive
potential. \cite{zala2004abnormal} presents a useful review of the
the effects of behaviour disrupting contaminants in a broad range of
animals.

Further complications arise when components of a system experience
fundamental changes in their basic dynamics, such as might occur in a
population with adults which care for dependent young, or situations
where adaptive behaviour is reinforced or breaks down as a consequence
of environmental change.  This situation is particularly prone to
``switching'' behaviour; populations of animals, for example, may have
quite different behaviours according to their population density; a
seminal, (and grim) example of this is described in
\citet{calhoun1973death}. A less maladaptive example, concerning the
behaviour of lynxes in response to declining prey populations, can be
found in \citet{ward1985behavioural}; in this paper Ward notes that
some animals choose a nomadic lifestyle when prey is scarce as a means
of optimising their likelihood of hunting success, while others do not.
In these cases, there are no pathogens or chemicals introduced which
alter the behaviour of the organisms.

Systems comprised of a number of the types of these types of systems
may exhibit dynamics which are difficult to address in a conventional
way. Moreover, the complexity of code intended to compass the range of
possible modal states may become a source of errors or artifacts in
the generated data. It may be difficult to validate a model which has
many possible concurrent execution paths. While many modelling systems
encourage loose coupling and narrow interfaces between components, the
complexity and size of the domains being modelled is steadily
increasing.  Even with very clean, robust code, the probability of
poor interactions during a run increases dramatically as size of
submodels and number of types and instances of these components
increases. In my own experience, the size and complexity of the
domains modelled has been increasing by roughly a factor of four every
seven years, though it may be much faster.

The basic questions, then, are ``\emph{How do we deal with situations 
where the assumptions that underpin our representation no longer hold?}'' 
and ``\emph{How do we manage the execution of submodels which simulate
systems or entities with multi-modal behaviour?}''

This thesis puts forward the argument that better models may be built
if we allow the representation of component parts of a model to change
according to their own needs and state, and the needs and states both
of their neighbours and of the model as a whole. Such a strategy has a
number of potential benefits,
\begin{itemize}
\item we can make many simple representations for a submodel, each of which
      deals well with a particular part of the submodel's domain,
\item the comparative simplicity of these representations effectively 
      reduces the number of potential code paths within a model at any
      given moment, since representations do not have to cope with
      edge cases, they merely indicate that they are entering a marginal
      or inappropriate domain
\item we can use analytic representations which are more efficient at
      representing large numbers of entities
\item we can use individual-based representations which capture
      the fine-scale dynamics that dominate when we are dealing with
      discrete events or low numbers of entities
\item we can choose representations that make the best use of available
      data within the model, or can ask for better representations
      in the ensemble
\item it is simple to incorporate code to track information about
      representation changes, relative execution speed, and cumulative
      error into the modelling system
\item we can include (or not) agents that identify the emergence of 
      perverse dynamics within the system.
\item we can decouple the production of the results from processes
      which simulate the systems and subsystems being modelled
\end{itemize}


In both of these questions, we can address the issue by explicitly
changing the representation of the entity in question.  So, rather
than a single representation for the population of a coastal city, we
may have a number of different \emph{submodels} with different levels
of aggregation and different temporal or spatial scales.  In a
hypothetical simulation of a cyclone season, we may represent the
population in its most aggregated form until bad weather starts
forming at sea.  As a tropical storm builds, we may dis\-aggregate the
population into \emph{agents} --- instances of submodels --- which
represent groups that live a long way inland, and other groups which
may be more at risk.  As a cyclone approaches, we may resolve the
representation further, instantiating emergency response agents and
individuals at risk.  In the aftermath, aggregation may occur in
regions where there are no acute effects, but other parts of the
system may remain finely resolved.

In this example, we can change the representations to deal with both
of the questions above: We initially assumed that for most of the
purposes of the simulation, our population could be treated as
relatively homogeneous.  We may have kept aggregate information about
population distribution, wealth and demographic characteristics, but
as soon as the simulation needed to ``mark'' only a portion of the
population (those with damaged property, for example), we would have
to refine our representation.  Similarly, the behaviours following the
storm are mod-ally different: those who live in protected areas are
mostly free to carry on with essentially the same ``normal''
representation, but those who lived in damaged areas must engage in
quite different activities, and may have quite a different exposure to
risks.

Suppose now that the damage has been repaired and we wish to
consolidate our representations and return to the original
representation for the city. Exploring this facet is a major component
of the model presented in Chapter \ref{modelefficiency}



% SE-jarosite: 6 agent types (squid, jack mackrel, tuna, anson, plume,
% current) ~200 agents

% NWS: ~46 (23 top level types used in NWS) agent types, ~800 
% Ningaloo: ~170 agent types with agent count in the 1000-2000 range

% The primary aim of this thesis is to demonstrate an approach to
% constructing computational models which can be used to effect changes
% in the mathematical or algorithmic structure of model of complex
% systems in response to changes in their status.


% Traditionally, we might address changes in the dynamcis of a system by
% incorporating alternative code paths or conditional expressions within
% a model

% Factors, such as the increasing need to explore the range of
% consequences of management decisions and environmental changes, 

% whether the environment we are talking about is the biotic/abiotic
% world around us, the social support network for disadvantaged people
% or the supply chain for a manufacturer 





% The primary aim of this thesis is to demonstrate an approach to
% constructing computational models which can be used to effect changes
% in the mathematical or algorithmic structure of model of complex
% systems in response to changes in their status. A simple
% uptake-depuration model of organisms which migrate through a
% contaminant field is presented in Chapter \ref{modelefficiency} to
% explore the fundamental principles and to demonstrate the utility of
% such a mechanism. The third chapter develops a mathematical structure
% with properties that make it a useful mechanism for representing the
% complex states of these structurally adaptible models. A number of
% useful functions are defined, and the basic properties of the
% structure are proved. Chapter \ref{adaptiveselection} uses this mathematical
% structure for encoding the complex states of a model and its
% components in a ``thought experiment'' which illustrates the dynamics
% which might arise in a simple model. 

% Chapter \ref{explicitmodel} provides a detailed discussion of a real
% implementation of the thought model used in Chapter
% \ref{adaptiveselection}. The chapter briefly describes the modelling
% framework used to implement the model, and then addresses the details
% of the implementation. \textbf{Using ODD?} {\emph The body of code is included in
% the Appendix.  If we include it all it is 500pp}


In this thesis, we will use \emph{model} to refer to a (possibly
trivial) collection of of components which, together, simulate a
system. The term \emph{submodel} will generally refer to a largely
discrete component of a model (arguably a model in its own right).
More specifically, we will consider submodels to be the abstract or
\emph{in potentia} representations, and \emph{agents} to be the
\emph{in esse} instances of submodels.

Traditional models of complex systems incorporate alternative code
paths or expressions to deal with fundamentally different dynamics
which may be required within a component of the model by testing for
conditions which arise during a simulation and executing an alternate
code path within the component. Examples of this approach can be found
in models implemented in a number of popular modelling frameworks
\citep{netlogo,swarm,repast,mason} and in my previous work
\citep{nws,ningaloo}.

In contrast, the models used as examples in this thesis deal with such
changes by changing the component used --- a number of agents
(instances of submodels)\footnote{} associated with individual contact
with a contaminant plume might be substituted for a population agent
which has no code path of its own to deal with contaminant uptake or
depuration.

Representing the state of the model, either as a whole or of its
constituent parts, is not simple: not only may the components of the
model vary through time, but their own dependences on other components
may change as the state of the model changes: a model which contains a
``whale spotting'' tourism venture may follow the activities of whales
at particular times of the year, but be utterly indifferent to them at
times when there is little likelihood of their presence in an
accessible location. Similarly, the association of entities
represented by a population-based model needs to be maintained if the
population dis\-aggregates into agents based on super-individuals
(small cohorts) or agents representing individuals.

These factors make a simple vector-based encoding mechanism for the
states of a model and its components awkward, thus we turn to a metric
space over a \rng\footnote{To avoid confusion, we take \rng\ to denote
  a ring without a multiplicative identity element.} of trees with a
finite number of weighted, labelled nodes. This simplifies the
comparison of possible component mixes for a given global state, and
makes available any algorithms (such as for clustering) which depend
only on ring properties, excluding those that need a multiplicative
unit.

Any attempt to assess the relative merit of a number of possible
component mixes for a model relative to its current state must have
some way of incorporating the necessary dependencies arising from the
effects of the current state on the , and be able to calculate (in
some sense) the incremental costs of change.


\typeout{Chapter 1: Historical work}
\section{Historical work}

Many individual-based models incorporate environmental characteristics
that influence the behaviour of of the individuals
simulated. \citet{Botkin72:2} and \citet{deangelis1978model} are
important early examples, modelling individuals in in the context of
spatially explicit environmental gradients. In both cases, the
environments are relatively static (though interactions, such as
shading in JABOWA, also played a role).  Even as formulated, we can
consider the environments in these models as static submodels in their
own right: from its early inception, the individual-based modelling
approach implicitly couples different styles of model components.

The coastal marine ecosystem models in \cite{Gray06:1} used different
representations for the organisms based on their life-stage. As
organisms that comprised the benthic habitat matured their
representations would change to suit their niche in the system.  In
this case the sequence of transitions was determined before
compilation of the model; juvenile biomasses could be represented
either by gridded cellular automata or by polygonal clouds which
changed shape as they were advected, while adult stages could be
represented by several different submodels which might be optimised
either for speed or for spatial fidelity.  This model was in most
other ways similar to conventional agent-based modelling of the time.

%\cite{gross2002multimodeling} describes using this approach as a
%basic element underpinning their model, ATLSS, of the Everglades
%region in the U.S.A.

\cite{bobashev2007hybrid} describes a model of epidemic simulation in
which the representation of populations or portions of populations are
decided based on the number of infected individuals relative to a
nominated trigger value. This model demonstrates that there is an
advantage to changing representation in terms of computational
efficiency and the fidelity of the model. The model in Chapter
\ref{modelefficiency} is similar: the rule governing the switching
from one representation to another depends on the state of the system:
switching occurs when some monitored quantity crosses a nominated
boundary.  The primary operational model underpinning the work in
\cite{Gray2014,Fulton2009crossingscales} bases its switching decision on the
location of individual whales, migrating whales are effectively
removed from the system while they are outside the domain of the
model, maintaining them in a rudimentary way until it was time for
them to migrate back into the model domain. Other agents within the
model, predators, wildlife management and tourism operators, were
influenced in their decision processes by the presence or absence of
whales in the model domain.

\typeout{Chapter 1: Structure and function}
\section{Structure and function}

Replacing an instance of a submodel representing some entity within a
model may substantially alter the natural time steps or spatial scales
associated with an entity in the simulation model: an individual-based
representation is likely to require a much shorter time step and a much
richer spatial environment than a population-based representation, for
example.  In practice, changing spatial resolution is relatively
straightforward --- either the data is there already, we interpolate
or we extrapolate. The sorts of causal issues that accompany
predation, for example, tend not to arise when we move from a finely
gridded landscape to a coarser version.

Deciding how to manage the flow of time in an ecological simulation
model is one of the first decisions in its design.  In decades past, a
model might be constructed as a large set of arrays containing state
variables which are inspected and updated in the body of a for
loop. Some models achieve temporal optimisation by dividing the arrays
into various groups of fast-stepping and slower-stepping variables,
only dealing with the necessary parts of the system at each time-step
(\emph{variable speed splitting} as in \cite{walters2000ecosystem},
for example). \cite{Gray06:1} and \cite{Gray2014} allow agents to
dynamically determine their own time step based on their state ---
time steps may be truncated, or changed for their next turn. There is a
trade-off in this, since with a variable speed splitting approach, we
can calculate all values based on a temporally coherent set of data,
and update them all in one pass, but, in the dynamic time-stepping
approach, each interaction is essentially conducted in isolation, and
the consequences of a set of interactions may be dependent on the
order in which the interactions occurs.

Both variable speed splitting and dynamic time-step selection are
flexible enough to support representational changes for entities









