% chap5.tex (Chapter 5 of the thesis)


\chapter[An Explicit Implementation]{An Explicit Implementation}
\WeAreOn{\cfive}\label{explicitmodel}

Chapter \ref{modelefficiency} argued that switching models were worth
considering on the basis of fidelity and efficiency. The results in
the chapter clearly demonstrate advantages, and while the example used
is quite simple compared to other models of marine or aquatic
contaminant interaction, the results are credible.  The contaminant
models I am most familiar with, \cite{Gray2006nws, Gray2014}, would
have benefited greatly from the approach. Both models flirted with the
idea without fully implementing it -- the latter came close, but
required an extra,  separate modelling kernel to maintain the whales
which left the system, rather than truly switching representations.
The source code for the model is available at \linebreak
\texttt{https://github.com/snarkypenguin/Explicit-Model.git}.

\section{Software environment for the model}
The model written to the template provided by
Chapter \ref{adaptiveselection} seeks to redress this absence.  While
arguably a model with very little real world application, the model
seeks to provide a framework and a template for constructing useful
models. The framework provided is based on Tiny-CLOS by Gregor
Kiczales while he was working at Xeros PARC in 1992 and 1993.
Tiny-CLOS is the basis from which many of the object systems used in modern
implementations of Scheme were derived or designed.  I have used
Gambit-C, a well regarded implementation of Scheme which is both
compilable and readily links (even uncompiled) to external code and
libraries. The choice was heavily influenced by the fact that it
compiled to acceptibly fast code, and that adding hand-crafted C or
C++ code for particularly intensive routines was not difficult.  
A lesser consideration was that Gambit-C runs on all of the major
platforms (Linux, Unix, MacOS, and Android, for example).
Porting the model to other versions of Scheme should be relatively
simple; the only potentially awkward issue would be the use
of \texttt{define-macro} which is used to make the process of defining
and using objects simpler and less error prone.

\section{Model overview}
How the model is organised
\begin{itemize}
\item The framework file and framework.scm
\item local overloading of (load)
\item overview of the classes
\item how the files are connected (need to make it \textbf{not}
dependent on the local initialisation.
\end{itemize}

\section{The Kernel}
how the kernel works, call-backs into the kernel

\section{Agents}
what makes an agent an agent, and why it is different from other object
derived things

\subsection{Environment classes}
Describe the basic environment classes and why they are there

\subsection{Plant classes}
Plants: pertinent eh?

\subsection{Animal classes}
Difference between herbivores and carnivores

\subsection{Data Loggers}
How the loggers work with their call-backs
