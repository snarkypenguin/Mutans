% app0.tex (file to switch to appendix mode)
% No need to alter this file...
\appendix
\chapter[APPENDIX A]{Appendix A}\label{appixa}
Some of the syntactic attributes of code in the example implementation
are represented with particular print styles, specifically

\begin{table}[H]
\begin{center}
  \caption{Printing styles\label{printstyles}}
  \begin{tabular}{l L{7cm}}
    \toprule
    \textbf{Example} & \textbf{role} \cr
    \midrule
    \Syntax{scheme-syntax} & indicates a syntactic component
      of \Scheme, or a macro declared in \fname{framework}\cr
    \hline & \cr
    \symbolic{symbolic-value} & indicates a \Scheme symbol\cr
    \hline & \cr
    \strng{string} & indicates a string of characters\cr
    \hline & \cr
    \variable{varname} & a variable name\cr
    \hline & \cr
    \method{methodname} & the name of a method\cr
    \hline & \cr
    \sttvar{agent-state-variable} & a state variable associated with   an agent\cr
    \hline & \cr
    \mclass{classname} & an \SCLOS class.\cr
    \bottomrule
  \end{tabular}
\end{center}
\end{table}

The following are examples of the output produced by
a \mclass{log-map} agent.

\begin{figure}\label{timesstepo}
\begin{center}
  \includegraphics[width=15cm, height=21cm]{T00}
  \caption{Initial state of the domainfor example run.}
  \end{center}
\end{figure}

\begin{figure}\label{timessteptf}
\begin{center}
  \includegraphics[width=15cm, height=21cm]{T25}
  \caption{The domain after 25 days.}
  \end{center}
\end{figure}

\begin{figure}\label{timesstepfo}
\begin{center}
  \includegraphics[width=15cm, height=21cm]{T50}
  \caption{The domain after 50 days.}
  \end{center}
\end{figure}

